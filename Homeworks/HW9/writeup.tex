\documentclass[12pt]{article}
\title{M368K Homework 9 \\
  \normalsize{\S~11.1 \#2a$^1$, 9\quad \S~11.2 \#4a$^2$}}
\author{Hershal Bhave (hb6279)}
\date{April 5, 2013}

\usepackage{mathtools}
\usepackage{amsfonts}
\usepackage{mathabx}
\usepackage{listings}
\usepackage[in]{fullpage}
\usepackage{color}
\usepackage{tabularx}
\usepackage{caption}
\usepackage{cleveref}
\usepackage{titlesec}
\usepackage{float}
\usepackage{caption}
\usepackage{subcaption}

\definecolor{mygreen}{rgb}{0,0.6,0}
% \definecolor{mygreen}{rgb}{0.13,0.55,0.13}
\definecolor{mygray}{rgb}{0.5,0.5,0.5}
\definecolor{mymauve}{rgb}{0.58,0,0.82}

\lstset{
  backgroundcolor=\color{white},
  basicstyle=\scriptsize\ttfamily,
  breakatwhitespace=false,
  breaklines=true,
  captionpos=b,
  commentstyle=\color{mygreen},
  deletekeywords={...},
  escapeinside={\%*}{*)},
  extendedchars=true,
  frame=single,
  keywordstyle=\color{blue},
  % language=Octave,
  % numbers=left,
  % numbersep=5pt,
  % numberstyle=\tiny\color{mygray},
  rulecolor=\color{black},
  showspaces=false,
  showstringspaces=false,
  showtabs=false,
  % stepnumber=2,
  stringstyle=\color{mymauve},
  tabsize=2,
  title=\lstname,
  columns=fullflexible,
}

\begin{document}
\maketitle

\section{\S~11.1}
\subsection{2a$^1$}
The boundary value problem 
\begin{equation}
  \label{eq:11.1_2a}
  y''=y'+2y+\cos x,\quad 0\leq x \leq 1,\quad y(0)=-0.3,\quad y(1)=-0.1
\end{equation}
has the solution $y(x)=-\frac{1}{10}(\sin x + 3\cos x)$. Use the
Linear Shooting method with $h=\frac{\pi}{4}$ to approximate the
solution to \cref{eq:11.1_2a} and compare the results to the actual
results. Also find an approximate solution using one step of
Secant-Euler with
$N=2,\,t_0=\frac{\beta-\alpha}{b-a},\,t_1=1.1t_0$. Compare the
approximate solution associated with $t_2$ to the exact solution at
each node.

\subsection{9}
Consider the boundary value problem 
\begin{equation}
  \label{eq:11.1_9}
  y''+y=0\quad, 0\leq x\leq b,\quad y(0)=0,\quad y(b)=B.
\end{equation}
Find choices for $b$ and $B$ so the boundary value problem has
\begin{enumerate}
\item No Solution
\item Exactly One Solution
\item Infinitely Many Solutions
\end{enumerate}

\Cref{eq:11.1_2a} is a second-order linear differential equation,
and can be solved as such. Assume $y=e^{rx}$, then we can see that

\begin{equation*}
  \begin{aligned}
    y''+y&=0 \\
    r^2e^{rx}+e^{rx}&=0 \\
    r^2+1&=0 \\
  \end{aligned}
\end{equation*}
Which implies that
\begin{equation}
  \label{eq:11.1_9_y}
  y(x)=c_1\cos x + c_2\sin x
\end{equation}
From \cref{eq:11.1_9_y} we can evaluate the first boundary condition
$y(0)=0$.
\begin{equation}
  \label{eq:11.1_9_c1}
  \begin{aligned}
    y(x)&=c_1\cos x + c_2\sin x \\
    y(0)&=c_1\cos 0 + c_2\sin 0 \\
    \implies c_1&=0 \\
  \end{aligned}
\end{equation}

Now we can find values of $b$ and $B$ for the second boundary
condition for which the boundary value problem has the end results
specified in \cref{eq:11.1_9_c1}. We will first consider the case when
there is no solution. Assuming $y(0)=0$, consider the second condition
to be $y(2\pi)=3$. Then we will obtain the following.

\begin{equation}
  \label{eq:11.1_9_nosol}
  \begin{aligned}
    y(2\pi)&=c_1\cos 2\pi + c_2\sin 2\pi \\
    3&=c_1\cos 2\pi + c_2\sin 2\pi \\
    \implies c_1 &= 3 \\
  \end{aligned}
\end{equation}
We have already mentioned from \cref{eq:11.1_9_c1} that $c_1=0$, but
we have obtained that $c_1=3$ from \cref{eq:11.1_9_nosol}.  This
discrepancy means that for the boundary condition $y(2\pi)=3$ there is
no solution for the system.
\begin{equation*}
\boxed{
  \begin{array}[c]{c}
    \text{No Solution:} \\
    b=2\pi,\,B=3 \\
  \end{array}
}
\end{equation*}


Now examine the case where there are infintely many solutions.
Assuming $y(0)=0$, consider the second condition to be
$y(2\pi)=0$. Then we will obtain the following.
\begin{equation}
  \label{eq:11.1_9_infsol}
  \begin{aligned}
    y(2\pi)&=c_1\cos 2\pi + c_2\sin 2\pi \\
    0&=c_1\cos 2\pi + c_2\sin 2\pi \\
    \implies c_1 &= 0 \\
  \end{aligned}
\end{equation}
We have already mentioned form \cref{eq:11.1_9_c1} that $c_1=0$, so
for the boundary condition $y(2\pi)=0$ there are infinitely many
solutions since we only get the same information about $c_1$ from
the computation involved in \cref{eq:11.1_9_infsol}.
\begin{equation*}
\boxed{
  \begin{array}[c]{c}
    \text{Infinitely Many Solutions:} \\
    b=2\pi,\,B=0 \\
  \end{array}
}
\end{equation*}


Finally, lets consider the case where there is a solution. If we
consider the second condition to be $y(\pi)=0$, then the following will
result.
\begin{equation}
  \label{eq:11.1_9_onesol}
  \begin{aligned}
    y(\pi)&=c_1\cos \pi + c_2\sin \pi \\
    0&=c_1\cos \pi + c_2\sin \pi \\
    \implies c_2 &= 0 \\
  \end{aligned}
\end{equation}
Since we have a value for $c_1$ from \cref{eq:11.1_9_c1} and we have
found a unique $c_2$ from \cref{eq:11.1_9_onesol}, then we are sure
that there is only one solution to the boundary value problem which
can satisfy both of these conditions.
\begin{equation*}
\boxed{
  \begin{array}[c]{c}
    \text{One Solution:} \\
    b=\pi,\,B=0 \\
  \end{array}
}
\end{equation*}

\section{\S~11.2}
\subsection{4a$^2$}
Use the Nonlinear Shooting method with $TOL=10^{-4}$ and $h=0.1$ to
approximate the solution to the boundary value problem in
\cref{eq:11.2_4a}.  The actual solution is given for comparison to
your results. Also find an approximate solution using one step of
Secant-Euler with
$N=2,\,t_0=\frac{\beta-\alpha}{b-a},\,t_1=1.1t_0$. Compare the
approximate solution associated with $t_2$ to the exact solution at
each node.
\begin{equation}
  \label{eq:11.2_4a}
  y''=y^3-y\,y',\quad 1\leq x\leq 2,\quad y(1)=\frac{1}{2},\quad 
  y(2)=\frac{1}{3}
\end{equation}

\section{Programming Minilab}

\end{document}