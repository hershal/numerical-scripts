\documentclass[12pt]{article}
\title{M368K Homework 13 \\
  \normalsize{\S~12.3 \#2$^1$,\quad \S~12.4 \#1$^2$ }}
\author{Hershal Bhave (hb6279)}
\date{Due May 3rd, 2013}

\usepackage{mathtools}
\usepackage{amssymb}
\usepackage{amsfonts}
\usepackage{mathabx}
\usepackage{listings}
\usepackage[in]{fullpage}
\usepackage{color}
\usepackage{tabularx}
\usepackage{caption}
\usepackage{cleveref}
\usepackage{titlesec}
\usepackage{float}
\usepackage{caption}
\usepackage{subcaption}
\usepackage{wrapfig}
\usepackage{cancel}
\usepackage{multirow}
\usepackage{tikz}
\usepackage{tikzscale}
\usepackage{gnuplot-lua-tikz}

\newcommand{\dd}[1]{\mathrm{d}{#1}}
\newcommand{\ddt}[1]{\frac{\dd{}}{\dd{#1}}}
\newcommand{\dddt}[1]{\frac{\dd{}^2}{\dd{#1}^2}}
\newcommand{\aoneonematrix}{
  \begin{bmatrix}
    1 & 0.25 & 0.75 \\
    1 & 0    & 0.5 \\
    1 & 0    & 1 \\
  \end{bmatrix}
}
\newcommand{\aonetwomatrix}{
  \begin{bmatrix}
    1 & 0.25 & 0.75 \\
    1 & 0    & 0.5  \\
    1 & 0.5  & 0.5  \\
  \end{bmatrix}
}
\newcommand{\bonematrix}[2]{
  \begin{bmatrix}
    a_{#1}^{(#2)} \\
    b_{#1}^{(#2)} \\
    c_{#1}^{(#2)} \\
  \end{bmatrix}
}
\newcommand{\setupaonematrices}[2]{
\aoneonematrix \bonematrix{#1}{#2}
}
\newcommand{\setupatwomatrices}[2]{
\aonetwomatrix \bonematrix{#1}{#2}
}

\definecolor{mygreen}{rgb}{0,0.6,0}
\definecolor{mygray}{rgb}{0.5,0.5,0.5}
\definecolor{mymauve}{rgb}{0.58,0,0.82}

\lstset{
  backgroundcolor=\color{white},
  basicstyle=\scriptsize\ttfamily,
  breakatwhitespace=false,
  breaklines=true,
  captionpos=b,
  commentstyle=\color{mygreen},
  deletekeywords={...},
  escapeinside={\%*}{*)},
  extendedchars=true,
  frame=single,
  keywordstyle=\color{blue},
  % language=Octave,
  % numbers=left,
  % numbersep=5pt,
  % numberstyle=\tiny\color{mygray},
  rulecolor=\color{black},
  showspaces=false,
  showstringspaces=false,
  showtabs=false,
  % stepnumber=2,
  stringstyle=\color{mymauve},
  tabsize=2,
  title=\lstname,
  columns=fullflexible,
}

\relpenalty=10000
\binoppenalty=10000

\begin{document}
\maketitle

\section{\S~12.3}
\subsection{2$^1$}
Approximate the solution to the given wave equation (hyperbolic
partial differential equation) by using the Central-Difference Method
with $\Delta x = \frac{1}{8}$ and $\Delta t= \frac{1}{10}$. Compute up
to $t=\frac{3}{10}$ and compare approximate and exact solutions at
each node at the final time.

The actual solution is $u(x,t)=\sin t \sin 4\pi x$.
\begin{equation}
  \label{eq:2_q}
  \begin{aligned}
  \frac{\delta^2u}{\delta t^2} - \frac{1}{16\pi^2}
  \frac{\delta^2u}{\delta x^2}  = 0, & \quad 0<x<1,\quad 0<t; \\
  u(0,t) = u(1,t) = 0, & \quad 0<t \\
  u(x,0) = \sin \pi x, & \quad 0\leq x \leq 1, \\
  \frac{\delta u}{\delta t}(x,0) = 0, & \quad 0\leq x \leq 1,
  \end{aligned}
\end{equation}

From the given equation we have the following constraints:
\begin{equation}
  \label{eq:2_const}
  \begin{aligned}
    h &= \Delta x = \frac{1}{8} \qquad&  \alpha^2 &= \frac{1}{16\pi^2} \\
    k &= \Delta t = \frac{1}{10} \qquad& \lambda &= \alpha(k/h) = \frac{1}{5\pi} \\
  \end{aligned}
\end{equation}

Now we can construct the A matrix and initial $\mathbf{w}$ vectors
using $\lambda$, $u(x,0)$, and $\frac{\delta u}{\delta t}(x,0)$.

\begin{equation}
  \label{eq:2_A}
  \begin{aligned}
    A &=
    \begin{pmatrix}
%%% BEGIN NO FILL
      2(1-\lambda^2) & \lambda^2    & 0       & \multicolumn{2}{c}{\cdots} & 0 \\
      \lambda^2    & 2(1-\lambda^2) & \lambda^2 & \ddots &  & \multirow{2}{*}{\vdots} \\
      0 & \ddots & \ddots & \ddots & \ddots & & \\
      \multirow{2}{*}{\vdots} & \ddots & \ddots & \ddots & \ddots & 0 \\
      & & \ddots & \lambda^2 & 2(1-\lambda^2) & \lambda^2 \\
      0 & \multicolumn{2}{c}{\cdots} & 0 & \lambda^2 & 2(1-\lambda^2) \\
%%% END NO FILL
    \end{pmatrix} \\
    \\
    \mathbf{w}^{(0)} &= f(x_i) \\
    \mathbf{w}^{(1)} &=
    (1-\lambda^2)f(x_i)+\frac{\lambda^2}{2}f(x_{i+1}) +
    \frac{\lambda^2}{2}f(x_{i-1}) + kg(x_i)
  \end{aligned}
\end{equation}
The Discrete Equations for this method are 
\begin{equation}
  \label{eq:2_dis}
  w_{i,j+1} = 2(1-\lambda^2)w{i,j} +
  \lambda^2(w_{i+1,j}+w_{i-1,j})-w_{i,j}-1 
\end{equation}
which turns out to be
\begin{equation}
  \label{eq:2_dis_val}
  w_{i,j+1} = 2(1-\frac{1}{5\pi}^2)w{i,j} +
  \frac{1}{5\pi}^2(w_{i+1,j}+w_{i-1,j})-w_{i,j}-1 
\end{equation}
where the boundary conditions imply that
\begin{equation}
  \label{eq:2_bound}
  w_{0,j} = w_{m,j} = 0.
\end{equation}
And now further approximations of $\mathbf{w}$ can be computed by
\begin{equation}
  \label{eq:2_wnext}
  \mathbf{w}^{(j+1P)} = A\mathbf{w}^{(j)} - \mathbf{w}^{(j-1)}
\end{equation}
Higher orders of $\mathbf{w}$ can be obtained by simple matrix
multiplication and then subtraction, displayed in \cref{eq:2_wnext}.

Using the algorithm given in \cref{lst:wavediff}, I obtained the data
in \cref{tab:2_1,tab:2_2}

\begin{table}[H]
  \centering
  \begin{tabular}[p]{cccc}
    \hline
    $x_i$ & $u(x_i,0.3)$ & $w(i,0.3)$ & $|u(x_i,0.3)-w(i,0.3)|$ \\
    \hline
    0 & 0 & 0 & \\
    0.125 &  0.29676 &  0.29676 & 1.2441$\times 10^{-3}$ \\
    0.250 &  0 & 0 \\
    0.375 & -0.29676 & -0.29676 & 1.2441$\times 10^{-3}$ \\
    0.500 &  0 & 0 \\
    0.625 &  0.29676 &  0.29676 & 1.2441$\times 10^{-3}$ \\
    0.750 &  0 & 0 \\
    0.875 & -0.29676 & -0.29676 & 1.2441$\times 10^{-3}$ \\
    1.000 &  0 & 0 & \\
    \hline
  \end{tabular}
  \caption{Data for Number 2 for t=3/10}
  \label{tab:2_1}
\end{table}

\begin{table}[H]
  \centering
  \begin{tabular}[p]{cccc}
    \hline
    $x_i$ & $u(x_i,0.5)$ & $w(i,0.5)$ & $|u(x_i,0.5)-w(i,0.5)|$ \\
    \hline
    0 & 0 & 0 & 0 \\
    0.125 &  0.48393 &  0.47943 & 4.5006 $\times 10^{-3}$ \\
    0.250 &  0.00000 &  0.00000 & 0 \\
    0.375 & -0.48393 & -0.47943 & 4.5006 $\times 10^{-3}$ \\
    0.500 &  0.00000 & -0.00000 & 0 \\
    0.625 &  0.48393 &  0.47943 & 4.5006 $\times 10^{-3}$ \\
    0.750 &  0.00000 &  0.00000 & 0 \\
    0.875 & -0.48393 & -0.47943 & 4.5006 $\times 10^{-3}$ \\
    1.000 & 0 & 0 & 0 \\
    \hline
  \end{tabular}
  \caption{Data for Number 2 for t=5/10}
  \label{tab:2_2}
\end{table}

\section{\S~12.4}
\subsection{1$^2$}
Use the Finite Element Method to approximate the solution to the partial
differential equation. Find the finite-element basisfunctions, arrays,
and approximate solution; Report the approximate solution at each
node. Let $M=2$; $T_1$ have vertices $(0,0.5), (0.25,0.75), (0,1)$; and
$T_2$ have vertices $(0,0.5), (0.5,0.5), (0.25,0.75)$.
\begin{equation}
  \label{eq:1_q}
  \begin{aligned}
    u_{xx}+4u_{yy} &= 3 &&\quad\text{ in } D, \\
    u(x,0.5) &= 2x
    &&\quad\text{ and } \\
    u(0,y) &= 0
    &&\quad\text{ on } \mathcal{S}_1, \\
    u_{x}\cos\theta_1 + 4u_{y}\cos\theta_2 &= (y-x)\frac{\sqrt{2}}{2}
    &&\quad\text{ on } \mathcal{S}_2.
  \end{aligned}
\end{equation}

%
% Note that for this problem,
% p = 1
% q = 4
% r = 0
% f = 3
% g = 2x and 0 (boundary conditions)
% g_1 = 0
% g_2 = (y-x)\frac{\sqrt{2}}{2}
%

We first divide the region $D$ into triangles $T_1$ and $T_2$
represented by \cref{fig:tri}.

\begin{figure}[H]
  \centering
  \includegraphics[width=.75\textwidth]{triangles.tikz}
  % \Huge{TODO: PUT THIS BACK IN WHEN DONE}
  \caption{Triangles for Number 1}
  \label{fig:tri}
\end{figure}
In this problem, we have 
\begin{equation}
  \label{eq:1_v}
  \begin{aligned}
    V_1^{(1)} &= (0.25,0.75) &\qquad V_1^{(2)} &= (0.25,0.75) \\
    V_2^{(1)} &= (0,0.5)     &\qquad V_2^{(2)} &= (0,0.5) \\
    V_3^{(1)} &= (0,1)       &\qquad V_3^{(2)} &= (0.5,0.5) \\
  \end{aligned}
\end{equation}
which we can write for simplicity as
\begin{equation}
  \label{eq:1_e}
  \begin{aligned}
    E_1 &= V_1^{(1)} = V_1^{(2)} &&= (0.25,0.75) \\
    E_2 &= V_2^{(1)} = V_2^{(2)} &&= (0,0.5) \\
    E_3 &= V_3^{(1)} &&= (0,1) \\
    E_4 &= V_3^{(2)} &&= (0.5,0.5). \\
  \end{aligned}
\end{equation}

We must now find the elements of the matrix $A = (\alpha_{i,j})$, for
$i=1,\ldots,n$ and $j=1,\ldots,m$, is in the form
\begin{equation}
  \label{eq:1_alpha_phi}
  \begin{aligned}
    \alpha_{i,j} &=
    % region D
    \int\int_{D}\left[
      p \,
      \frac{\partial\phi_i}{\partial x} \,
      \frac{\partial\phi_j}{\partial x} +
      q \,
      \frac{\partial\phi_i}{\partial y} \,
      \frac{\partial\phi_j}{\partial y} -
      r
      \phi_i
      \phi_i
    \right]
    \dd{x}\,\dd{y} +
    % region S_2 (why S_2?)
    \int_{\mathcal{S}_2}
      g_1\phi_i\phi_j \,
    \dd{\mathcal{S}_2} \\
  \end{aligned}
\end{equation}
and $\beta_i$, for $i=1,\ldots,n$, in the form
\begin{equation}
  \label{eq:1_beta_phi}
  \beta_i = -\int\int_{\mathcal{D}}f\phi_i\,\dd{x}\,\dd{y} +
  \int_{\mathcal{S}_2}g_2\phi_i\,\dd{\mathcal{S}} -
  \sum_{k=n+1}^{m}\alpha_{ik}\gamma_{k}
\end{equation}
where $\phi(x,y) = a + bx + cy$ for triangles.
For triangle $T_1$ we have
\begin{equation}
  \label{eq:1_t1_phi}
  \begin{aligned}
    % 
    \phi_1(x,y) &= N_1^{(1)}(x,y) =
    a_1^{(1)} + b_1^{(1)}x + c_1^{(1)}y,\\ 
    % 
    \phi_2(x,y) &= N_2^{(1)}(x,y) =
    a_2^{(1)} + b_2^{(1)}x + c_2^{(1)}y, \\
    % 
    \phi_3(x,y) &= N_3^{(1)}(x,y) =
    a_3^{(1)} + b_3^{(1)}x + c_3^{(1)}y,\\ 
    % 
    \phi_4(x,y) &= 0 \\
  \end{aligned}
\end{equation}
which implies that the partials for $\phi$ for $T_1$ are
\begin{equation}
  \label{eq:1_t1_dphi}
  \begin{aligned}
    \frac{\partial\phi_1}{\partial x} = b_1^{(1)},&\quad
    \frac{\partial\phi_1}{\partial y} = c_1^{(1)},&\quad
    \frac{\partial\phi_2}{\partial x} = b_2^{(1)},&\quad
    \frac{\partial\phi_2}{\partial y} = c_2^{(1)},&\quad \\
    \frac{\partial\phi_3}{\partial x} = b_3^{(1)},&\quad
    \frac{\partial\phi_3}{\partial y} = c_3^{(1)},&\quad
    \frac{\partial\phi_4}{\partial x} = 0,& \quad   
    \frac{\partial\phi_4}{\partial y} = 0.& \quad
  \end{aligned}
\end{equation}
and for triangle $T_2$ we have
\begin{equation}
  \label{eq:1_t2_phi}
  \begin{aligned}
    % 
    \phi_1(x,y) &= N_1^{(2)}(x,y) =
    a_1^{(2)} + b_1^{(2)}x + c_1^{(2)}y,\\ 
    % 
    \phi_2(x,y) &= N_2^{(2)}(x,y) =
    a_2^{(2)} + b_2^{(2)}x + c_2^{(2)}y,\\ 
    % 
    \phi_3(x,y) &= 0 \\
    % 
    \phi_4(x,y) &= N_3^{(2)}(x,y) =
    a_3^{(2)} + b_3^{(2)}x + c_3^{(2)}y,\\
  \end{aligned}
\end{equation}
which implies that the partials for $\phi$ for $T_2$ are
\begin{equation}
  \label{eq:1_t2_dphi}
  \begin{aligned}
    &\frac{\partial\phi_1}{\partial x} = b_1^{(2)},&\quad
    &\frac{\partial\phi_1}{\partial y} = c_1^{(2)},&\quad
    &\frac{\partial\phi_2}{\partial x} = b_2^{(2)},&\quad
    &\frac{\partial\phi_2}{\partial y} = c_2^{(2)},&\quad \\
    &\frac{\partial\phi_3}{\partial x} = 0,&\quad
    &\frac{\partial\phi_3}{\partial y} = 0,&\quad
    &\frac{\partial\phi_4}{\partial x} = b_3^{(2)},&\quad
    &\frac{\partial\phi_4}{\partial y} = c_3^{(2)}.&\quad
  \end{aligned}
\end{equation}

Each $N_j^{(i)}$ equation corresponds to the vertex $V_j^{(i)}$ and
produces systems in the form
\begin{equation}
  \label{eq:1_n_sys}
  \begin{bmatrix}
    1 & x_1^{(i)} & y_1^{(i)} \\
    1 & x_2^{(i)} & y_2^{(i)} \\
    1 & x_3^{(i)} & y_3^{(i)} \\
  \end{bmatrix}
  \begin{bmatrix}
    a_j^{(i)} \\ b_j^{(i)} \\ c_j^{(i)} \\
  \end{bmatrix}
  =
  \begin{bmatrix}
    j\stackrel{?}{=}k \\ j\stackrel{?}{=}k \\ j\stackrel{?}{=}k \\
  \end{bmatrix}
\end{equation}
and solving each of the systems gives
\begin{equation}
  \label{eq:1_A_array}
  \begin{aligned}
    % A_1^(1)
    \setupaonematrices{1}{1}&
    = \begin{bmatrix} 1\\0\\0\\\end{bmatrix};
    &\qquad\bonematrix{1}{1} &= \begin{bmatrix}0\\4\\0\\\end{bmatrix}\\
    % A_2^(1)
    \setupaonematrices{2}{1}&
    = \begin{bmatrix} 0\\1\\0\\\end{bmatrix};
    &\qquad\bonematrix{2}{1} &= \begin{bmatrix}2\\-2\\-2\\\end{bmatrix}\\
    % A_3^(1)
    \setupaonematrices{3}{1}&
    = \begin{bmatrix} 0\\0\\1\\\end{bmatrix};
    &\qquad\bonematrix{3}{1} &= \begin{bmatrix}-1\\-2\\2\\\end{bmatrix}\\
    % A_1^(2)
    \setupatwomatrices{1}{2}&
    = \begin{bmatrix} 1\\0\\0\\\end{bmatrix};
    &\qquad\bonematrix{1}{2} &= \begin{bmatrix}-2\\0\\4\\\end{bmatrix}\\
    % A_2^(2)
    \setupatwomatrices{2}{2}&
    = \begin{bmatrix} 0\\1\\0\\\end{bmatrix};
    &\qquad\bonematrix{2}{2} &= \begin{bmatrix}2\\-2\\-2\\\end{bmatrix}\\
    % A_3^(2)
    \setupatwomatrices{3}{2}&
    = \begin{bmatrix} 0\\0\\1\\\end{bmatrix}
    &\qquad\bonematrix{3}{2} &= \begin{bmatrix}1\\2\\-2\\\end{bmatrix}.\\
  \end{aligned}
\end{equation}

We can consider $E_1$, $E_2$, and $E_4$ as nodes on $\mathcal{S}_1$
where the boundary conditions
\begin{equation*}
  \begin{aligned}
    g_1(x,y) = u(x,0.5) &= 2x, \text{ and}\\
    g_2(x,y) = u(0,y) &= 0 \\
  \end{aligned}
\end{equation*}
are imposed. This implies
\begin{equation}
  \label{eq:1_gamma_2_3_4}
  \begin{aligned}
    \gamma_2 = 0, \gamma_3 = 0, \text{ and } \gamma_4 = 1.
  \end{aligned}
\end{equation}

Since we have already determined $\gamma_2$, $\gamma_3$, and $\gamma_4$,
we only need to find $\gamma_1$. To do this we need to consider
$\alpha_{1,1}$ for both triangles. So now the big integral for $\alpha$
in \cref{eq:1_alpha_phi} becomes
\begin{equation}
  \label{eq:1_alpha_1_1_nophi}
  \begin{aligned}
    \alpha_{1,1}&=
    b_1^{(1)}b_1^{(1)}\int\int_{T_1} p\, \dd{x}\,\dd{y} +
    c_1^{(1)}c_1^{(1)}\int\int_{T_1} q\, \dd{x}\,\dd{y} \\ &\qquad-
    \int\int_{T_1} r (a_1^{(1)} + b_1^{(1)}x + c_1^{(1)}y)(a_1^{(1)} +
    b_2^{(1)}x + c_2^{(1)}y) \dd{x}\,\dd{y} \\ &\quad+ 
    b_1^{(2)}b_1^{(2)}\int\int_{T_2} p\, \dd{x}\,\dd{y} +
    c_1^{(2)}c_1^{(2)}\int\int_{T_2} q\, \dd{x}\,\dd{y} \\ &\qquad-
    \int\int_{T_2} r(a_1^{(2)} + b_1^{(2)}x + c_1^{(2)}y)(a_1^{(2)} +
    b_1^{(2)}x + c_1^{(2)}y)\dd{x}\,\dd{y} \\
    &=16\int\int_{T_1} \, \dd{x}\,\dd{y} +
    64\int\int_{T_2} \, \dd{x}\,\dd{y} \\
    \alpha_{1,1}&=10
  \end{aligned}
\end{equation}
and now the $\beta$ equation from \cref{eq:1_beta_phi} becomes
\begin{equation}
  \label{eq:1_beta_prepara}
  \begin{aligned}
    \beta_1 &= -\int\int_{T_1}3\phi_1 \,\dd{x}\,\dd{y} -
    \int\int_{T_2}3\phi_1 \,\dd{x}\,\dd{y} \\ &\quad+
    \text{ proj}_{T_1}\int_{\mathcal{S}_2} 
    (y-x)\frac{\sqrt{2}}{2}\phi_1^{(1)}\,\dd{\mathcal{S}} +
    \text{ proj}_{T_2}\int_{\mathcal{S}_2}
    (y-x)\frac{\sqrt{2}}{2}\phi_1^{(2)}\,\dd{\mathcal{S}} \\ &\quad-
    \sum_{k=2}^{4}\alpha_{1,k}\gamma_{k}\\
  \end{aligned}
\end{equation}
We must parametrize the $\text{proj}_{T_i}$ entries in order to evaluate
the line integral. We do this by
\begin{equation}
  \label{eq:1_t1_para}
  \begin{aligned}
    \text{proj}_{T_1}\mathcal{S}_2 
    &:= <0.5-0.25t,\,0.5+0.25t>,&t\in(0,1) \\
    \text{proj}_{T_2}\mathcal{S}_2 
    &:= <0.25-0.25t,\,0.75+0.25t>,&t\in(0,1) \\
  \end{aligned}
\end{equation}
which makes 
\begin{equation}
  \label{eq:1_beta_para}
  \begin{aligned}
    \beta_1&= 0.375 +
    \int_0^1\left[((0.5-0.25t)-(0.5-0.25t))\frac{\sqrt{2}}{2}(4(0.5-0.25t))\right]\dd{t} 
    \\ &\quad+
    \int_0^1\left[((0.75-0.25t)-(0.25+0.25t))\frac{\sqrt{2}}{2}(-2+4(0.75+0.25t))\right]\dd{t}
    \\ &\quad- 2.5\\
    &= 0.375 + 3.53553 - 2.5 \\
    \beta_1 &= 1.4105
  \end{aligned}
\end{equation}
Now finally we can obtain $\gamma_4$ 
\begin{equation}
  \label{eq:1_gamma_4}
  (\alpha_{1,1})\gamma_4 = \beta_1 \Rightarrow \gamma_4 = 0.14105
\end{equation}
And finally have an approximation to the solution for \cref{eq:1_q}:
\begin{equation}
  \label{eq:1_approx_t_phi}
  \boxed{
    \begin{aligned}
      T_1:\phi(x,y)&=0.14105(4x), \\
      T_2:\phi(x,y)&=0.14105(-2+4y)+(1+2x-2y)
    \end{aligned}
  }
\end{equation}

\section{Programming Minilab}
\subsection{Part b}
\begin{table}[H]
  \centering
  \begin{tabular}[H]{ccc}
    \hline
    $i$ & $\Delta t_i$ & $u_i(0,2,7)$ \\
    \hline
    1 & $\frac{1}{10}$ & -7.02105 $\times 10^{23}$ \\
    2 & $\frac{1}{15}$ & 0.00586 \\
    3 & $\frac{1}{20}$ & 0.03561 \\
    4 & $\frac{1}{25}$ & 0.03558 \\
    5 & $\frac{1}{30}$ & 0.03557 \\
  \end{tabular}
  \caption{Data for Minilab Part a}
  \label{tab:mini_b}
\end{table}
The stability threshold was crossed after $\Delta t$ crossed
$\frac{1}{15}$, as the solution seems to stabilize after that spacing.
\subsection{Part c}
\begin{table}[H]
  \centering
  \begin{tabular}[H]{ccc}
    \hline
    $i$ & $t_i$ & $u_i(0,2,7)$ \\
    \hline
    1 & 3.5 & 0.00004 \\
    2 & 5   & 0.11079 \\
    3 & 7   & 0.03557 \\
  \end{tabular}
  \caption{Data for Minilab Part a}
  \label{tab:mini_b}
\end{table}

\pagebreak
\section{Code}
\lstinputlisting
[language=Octave,label=lst:wavediff,caption=\texttt{wavediff.m}]
{wavediff.m}
\pagebreak
\lstinputlisting
[language=C++,label=lst:p13,caption=\texttt{program13.cpp}]
{program13.cpp}
\pagebreak
\lstinputlisting
[language=C++,label=lst:ctrdiff2D,caption=\texttt{ctrdiff2D.cpp}]
{ctrdiff2D.cpp}

\end{document}