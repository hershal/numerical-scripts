\documentclass[12pt]{article}
\title{M368K Homework 13 \\
  \normalsize{\S~12.3 \#2$^1$,\quad \S~12.4 \#1$^2$ }}
\author{Hershal Bhave (hb6279)}
\date{Due May 3rd, 2013}

\usepackage{mathtools}
\usepackage{amssymb}
\usepackage{amsfonts}
\usepackage{mathabx}
\usepackage{listings}
\usepackage[in]{fullpage}
\usepackage{color}
\usepackage{tabularx}
\usepackage{caption}
\usepackage{cleveref}
\usepackage{titlesec}
\usepackage{float}
\usepackage{caption}
\usepackage{subcaption}
\usepackage{wrapfig}
\usepackage{cancel}
\usepackage{multirow}

\newcommand{\dd}[1]{\mathrm{d}{#1}}
\newcommand{\ddt}[1]{\frac{\dd{}}{\dd{#1}}}
\newcommand{\dddt}[1]{\frac{\dd{}^2}{\dd{#1}^2}}

\definecolor{mygreen}{rgb}{0,0.6,0}
\definecolor{mygray}{rgb}{0.5,0.5,0.5}
\definecolor{mymauve}{rgb}{0.58,0,0.82}

\lstset{
  backgroundcolor=\color{white},
  basicstyle=\scriptsize\ttfamily,
  breakatwhitespace=false,
  breaklines=true,
  captionpos=b,
  commentstyle=\color{mygreen},
  deletekeywords={...},
  escapeinside={\%*}{*)},
  extendedchars=true,
  frame=single,
  keywordstyle=\color{blue},
  % language=Octave,
  % numbers=left,
  % numbersep=5pt,
  % numberstyle=\tiny\color{mygray},
  rulecolor=\color{black},
  showspaces=false,
  showstringspaces=false,
  showtabs=false,
  % stepnumber=2,
  stringstyle=\color{mymauve},
  tabsize=2,
  title=\lstname,
  columns=fullflexible,
}

\relpenalty=10000
\binoppenalty=10000

\begin{document}
\maketitle

\section{\S~12.3}
\subsection{2$^1$}
Approximate the solution to the given wave equation (hyperbolic
partial differential equation) by using the Central-Difference Method
with $\Delta x = \frac{1}{8}$ and $\Delta t= \frac{1}{10}$. Compute up
to $t=\frac{3}{10}$ and compare approximate and exact solutions at
each node at the final time.

The actual solution is $u(x,t)=\sin t \sin 4\pi x$.
\begin{equation}
  \label{eq:2_q}
  \begin{aligned}
  \frac{\delta^2u}{\delta t^2} - \frac{1}{16\pi^2}
  \frac{\delta^2u}{\delta x^2}  = 0, & \quad 0<x<1,\quad 0<t; \\
  u(0,t) = u(1,t) = 0, & \quad 0<t \\
  u(x,0) = \sin \pi x, & \quad 0\leq x \leq 1, \\
  \frac{\delta u}{\delta t}(x,0) = 0, & \quad 0\leq x \leq 1,
  \end{aligned}
\end{equation}

From the given equation we have the following constraints:
\begin{equation}
  \label{eq:2_const}
  \begin{aligned}
    h &= \Delta x = \frac{1}{8} \qquad&  \alpha^2 &= \frac{1}{16\pi^2} \\
    k &= \Delta t = \frac{1}{10} \qquad& \lambda &= \alpha(k/h) = \frac{1}{5\pi} \\
  \end{aligned}
\end{equation}

Now we can construct the A matrix and initial $\mathbf{w}$ vectors
using $\lambda$, $u(x,0)$, and $\frac{\delta u}{\delta t}(x,0)$.

\begin{equation}
  \label{eq:2_A}
  \begin{aligned}
    A &=
    \begin{pmatrix}
      2(1-\lambda^2) & \lambda^2    & 0       & \multicolumn{2}{c}{\cdots} & 0 \\
      \lambda^2    & 2(1-\lambda^2) & \lambda^2 & \ddots &  & \multirow{2}{*}{\vdots} \\
      0 & \ddots & \ddots & \ddots & \ddots & & \\
      \multirow{2}{*}{\vdots} & \ddots & \ddots & \ddots & \ddots & 0 \\
      & & \ddots & \lambda^2 & 2(1-\lambda^2) & \lambda^2 \\
      0 & \multicolumn{2}{c}{\cdots} & 0 & \lambda^2 & 2(1-\lambda^2) \\
    \end{pmatrix} \\
    \\
    \mathbf{w}^{(0)} &= f(x_i) \\
    \mathbf{w}^{(1)} &=
    (1-\lambda^2)f(x_i)+\frac{\lambda^2}{2}f(x_{i+1}) +
    \frac{\lambda^2}{2}f(x_{i-1}) + kg(x_i)
  \end{aligned}
\end{equation}
The Discrete Equations for this method are 
\begin{equation}
  \label{eq:2_dis}
  w_{i,j+1} = 2(1-\lambda^2)w{i,j} +
  \lambda^2(w_{i+1,j}+w_{i-1,j})-w_{i,j}-1 
\end{equation}
which turns out to be
\begin{equation}
  \label{eq:2_dis_val}
  w_{i,j+1} = 2(1-\frac{1}{5\pi}^2)w{i,j} +
  \frac{1}{5\pi}^2(w_{i+1,j}+w_{i-1,j})-w_{i,j}-1 
\end{equation}
where the boundary conditions imply that
\begin{equation}
  \label{eq:2_bound}
  w_{0,j} = w_{m,j} = 0.
\end{equation}
And now further approximations of $\mathbf{w}$ can be computed by
\begin{equation}
  \label{eq:2_wnext}
  \mathbf{w}^{(j+1P)} = A\mathbf{w}^{(j)} - \mathbf{w}^{(j-1)}
\end{equation}
Higher orders of $\mathbf{w}$ can be obtained by simple matrix
multiplication and then subtraction, displayed in \cref{eq:2_wnext}.

Using the algorithm given in \cref{lst:wavediff}, I obtained the data
in \cref{tab:2_1,tab:2_2}

\begin{table}[H]
  \centering
  \begin{tabular}[p]{cccc}
    \hline
    $x_i$ & $u(x_i,0.3)$ & $w(i,0.3)$ & $|u(x_i,0.3)-w(i,0.3)|$ \\
    \hline
    0 & 0 & 0 & \\
    0.125 &  0.29676 &  0.29676 & 1.2441$\times 10^{-3}$ \\
    0.250 &  0 & 0 \\
    0.375 & -0.29676 & -0.29676 & 1.2441$\times 10^{-3}$ \\
    0.500 &  0 & 0 \\
    0.625 &  0.29676 &  0.29676 & 1.2441$\times 10^{-3}$ \\
    0.750 &  0 & 0 \\
    0.875 & -0.29676 & -0.29676 & 1.2441$\times 10^{-3}$ \\
    1.000 &  0 & 0 & \\
    \hline
  \end{tabular}
  \caption{Data for Number 2 for t=3/10}
  \label{tab:2_1}
\end{table}

\begin{table}[H]
  \centering
  \begin{tabular}[p]{cccc}
    \hline
    $x_i$ & $u(x_i,0.5)$ & $w(i,0.5)$ & $|u(x_i,0.5)-w(i,0.5)|$ \\
    \hline
    0 & 0 & 0 & 0 \\
    0.125 &  0.48393 &  0.47943 & 4.5006 $\times 10^{-3}$ \\
    0.250 &  0.00000 &  0.00000 & 0 \\
    0.375 & -0.48393 & -0.47943 & 4.5006 $\times 10^{-3}$ \\
    0.500 &  0.00000 & -0.00000 & 0 \\
    0.625 &  0.48393 &  0.47943 & 4.5006 $\times 10^{-3}$ \\
    0.750 &  0.00000 &  0.00000 & 0 \\
    0.875 & -0.48393 & -0.47943 & 4.5006 $\times 10^{-3}$ \\
    1.000 & 0 & 0 & 0 \\
    \hline
  \end{tabular}
  \caption{Data for Number 2 for t=5/10}
  \label{tab:2_2}
\end{table}

\section{\S~12.4}
\subsection{1$^2$}
Use the Finite Element Method to approximate the solution to the
partial differential equation. Find the finite-element basis
functions, arrays, and approximate solution; Report the approximate
solution at each node. Let $M=2$; $T_1$ have vertices $(0,0.5),
(0.25,0.75), (0,1)$; and $T_2$ have vertices $(0,0.5), (0.5,0.5),
(0.25,0.75)$. 
\begin{equation}
  \label{eq:1_q}
  \begin{aligned}
    u_{xx}+4u_{yy} &= 3 &\quad\text{ in } D, \\
    u(x,0.5) &= 2x, \\
    u(0,y) &= 0 &\quad\text{ on } S_1, \\
    u_{x}\cos\theta_1 + 4u_{y}\cos\theta_2 &= (y-x)\frac{\sqrt{2}}{2}
    &\quad\text{ on } S_2
  \end{aligned}
\end{equation}
\section{Programming Minilab}

\pagebreak
\section{Code}
\lstinputlisting
[language=Octave,label=lst:wavediff,caption=\texttt{wavediff.m}]
{wavediff.m}
\pagebreak
\lstinputlisting
[language=C++,label=lst:p13,caption=\texttt{program13.cpp}]
{program13.cpp}
\pagebreak
\lstinputlisting
[language=C++,label=lst:ctrdiff2D,caption=\texttt{ctrdiff2D.cpp}]
{ctrdiff2D.cpp}

\end{document}