\documentclass[12pt]{article}
\title{M368K Homework 12 \\
  \normalsize{\S~12.2 \#6$^1$, 8$^1$, 10$^1$ }}
\author{Hershal Bhave (hb6279)}
\date{Due April 26, 2013}

\usepackage{mathtools}
\usepackage{amssymb}
\usepackage{amsfonts}
\usepackage{mathabx}
\usepackage{listings}
\usepackage[in]{fullpage}
\usepackage{color}
\usepackage{tabularx}
\usepackage{caption}
\usepackage{cleveref}
\usepackage{titlesec}
\usepackage{float}
\usepackage{caption}
\usepackage{subcaption}
\usepackage{wrapfig}
\usepackage{cancel}
\usepackage{multirow}

\newcommand{\dd}[1]{\mathrm{d}{#1}}
\newcommand{\ddt}[1]{\frac{\dd{}}{\dd{#1}}}
\newcommand{\dddt}[1]{\frac{\dd{}^2}{\dd{#1}^2}}

\definecolor{mygreen}{rgb}{0,0.6,0}
\definecolor{mygray}{rgb}{0.5,0.5,0.5}
\definecolor{mymauve}{rgb}{0.58,0,0.82}

\lstset{
  backgroundcolor=\color{white},
  basicstyle=\scriptsize\ttfamily,
  breakatwhitespace=false,
  breaklines=true,
  captionpos=b,
  commentstyle=\color{mygreen},
  deletekeywords={...},
  escapeinside={\%*}{*)},
  extendedchars=true,
  frame=single,
  keywordstyle=\color{blue},
  % language=Octave,
  % numbers=left,
  % numbersep=5pt,
  % numberstyle=\tiny\color{mygray},
  rulecolor=\color{black},
  showspaces=false,
  showstringspaces=false,
  showtabs=false,
  % stepnumber=2,
  stringstyle=\color{mymauve},
  tabsize=2,
  title=\lstname,
  columns=fullflexible,
}

\relpenalty=10000
\binoppenalty=10000

\begin{document}
\maketitle

\section{\S~12.2}
\subsection{6$^1$}
\label{sec:6}
Use the Forward-Difference method to approximate the solution to the
following parabolic partial differential equations. Use $\Delta x =
\frac{1}{4}$ and $\Delta t = \frac{1}{10}$. Explicitly write the
discrete equations. Compute up to $t=\frac{2}{10}$; compare
approximate and exact solutions at each node at the final time.

\subsubsection{b}
\begin{equation}
\label{eq:6b_q}
\begin{aligned}
  \frac{\partial u}{\partial t} - \frac{1}{\pi^2}
  \frac{\partial^2u}{\partial x^2} &= 0,\quad 0<x<1,\quad 0<t;\\
  u(0,t) &= u(1,t)=0, \quad 0<t,\\
  u(x,0) &= \cos\left[\pi \left(x-\frac{1}{2}\right)\right], \quad
  0\leq x \leq 1.\\ 
\end{aligned}
\end{equation}

From the given equation we have the following constants:
\begin{equation}
  \label{eq:6b_const}
  \begin{aligned}
    h &= \Delta x = \frac{1}{4} \qquad&  \alpha^2 &= \frac{1}{\pi^2} \\
    k &= \Delta t = \frac{1}{10} \qquad& \lambda &= \alpha^2(k/h^2) = \frac{8}{5\pi^2} \\
  \end{aligned}
\end{equation}

Now we can construct the $A$ matrix and initial $\mathbf{w}$ vector using $\lambda$ and $u(x,0)$.
\begin{equation}
  \label{eq:6b_A}
  \begin{aligned}
    A &=
    \begin{pmatrix}
      % 1-2\lambda & \lambda & 0 & \multicolumn{3}{c}{\cdots} & 0 \\
      % \lambda & 1-2\lambda & \lambda & \multicolumn{2}{c}{\ddots} &
      % & \multirow{2}{*}{\vdots} \\
      % 0 & & \multirow{2}{*}{\ddots} \\
      % \vdots & & & & & & 0 \\
      % 0 & \multicolumn{2}{c}{\cdots} & 0 & \lambda & 1-2\lambda \\
      1-2\lambda & \lambda    & 0       & \multicolumn{2}{c}{\cdots} & 0 \\
      \lambda    & 1-2\lambda & \lambda & \ddots &  & \multirow{2}{*}{\vdots} \\
      0 & \ddots & \ddots & \ddots & \ddots & & \\
      \multirow{2}{*}{\vdots} & \ddots & \ddots & \ddots & \ddots & 0 \\
      & & \ddots & \lambda & 1-2\lambda & \lambda \\
      0 & \multicolumn{2}{c}{\cdots} & 0 & \lambda & 1-2\lambda \\
    \end{pmatrix} \\
    \\
    \mathbf{w}^{(0)} &= \sin\frac{\pi}{4}x \cdot
    \left(1+2\cos\frac{\pi}{4}x\right)\\
  \end{aligned}
\end{equation}

The discrete equations for the Forward-Difference method is
\begin{equation}
  \label{eq:6b_discrete}
  \mathbf{w}_{i,j+1}=(1-2\lambda)w_{i,j}+\lambda(w_{i+1,j}+w_{i-1,j})
\end{equation}
Which turns out to be 
\begin{equation}
  \label{eq:6b_discrete_val}
  \mathbf{w}_{i,j+1}=\left(1-\frac{16}{5\pi^2}\right)w_{i,j}+\frac{8}{5\pi^2}(w_{i+1,j}+w_{i-1,j})
\end{equation}
where 
\begin{equation}
  \label{eq:6b_w_i_0}
  \mathbf{w}_{i,0} = \cos\left[\pi\left(x_i-\frac{1}{2}\right)\right]
\end{equation}
And now further approximations of $\mathbf{w}$ can be computed by
\begin{equation}
  \label{eq:6b_wnext}
  \mathbf{w}^{(j)} = A\mathbf{w}^{(j-1)}
\end{equation}
As you can see, higher orders of $\mathbf{w}$ can be obtained by
simple multiplication in \cref{eq:6b_wnext}.

Using the algorithm given in \cref{lst:forwarddiff}, I obtained the
data in \cref{tab:6b_1,tab:6b_2}.

\begin{table}[H]
  \centering
  \begin{tabular}[p]{cccc}
    \hline
    $x_i$ & $u(x_i,1/10)$ & $w(i,1/10)$ & $|u(x_i,1/10)-w(i,1/10)|$ \\
    \hline
    0 & 0 & 0 & \\
    0.25 & 0.63982 & 0.63996 & 1.4033 $\times 10^{-4}$ \\
    0.50 & 0.90484 & 0.90504 & 1.9846 $\times 10^{-4}$ \\
    0.75 & 0.63982 & 0.63996 & 1.4033 $\times 10^{-4}$ \\
    1.00 & 0 & 0 & \\
    \hline
  \end{tabular}
  \caption{Data for Number 6b for t=1/10}
  \label{tab:6b_1}
\end{table}

\begin{table}[H]
  \centering
  \begin{tabular}[p]{cccc}
    \hline
    $x_i$ & $u(x_i,2/10)$ & $w(i,2/10)$ & $|u(x_i,2/10)-w(i,2/10)|$ \\
    \hline
    0 & 0 & 0 & \\
    0.25 & 0.57893 & 0.57918 & 2.5399 $\times 10^{-4}$ \\
    0.50 & 0.81873 & 0.81909 & 3.5919 $\times 10^{-4}$ \\
    0.75 & 0.57893 & 0.57918 & 2.5399 $\times 10^{-4}$ \\
    1.00 & 0 & 0 & \\
    \hline
  \end{tabular}
  \caption{Data for Number 6b for t=2/10}
  \label{tab:6b_2}
\end{table}

\subsection{8$^1$}
Repeat the question in \cref{sec:6} using the Backward-Difference Algorithm.

\subsubsection{b}

\Cref{eq:6b_const} still applies to this problem. We can construct the
$A$ matrix and initial $\mathbf{w}$ vector using $\lambda$ and $u(x,0)$.
\begin{equation}
  \label{eq:8b_A}
  \begin{aligned}
    A &=
    \begin{pmatrix}
      % 1-2\lambda & \lambda & 0 & \multicolumn{3}{c}{\cdots} & 0 \\
      % \lambda & 1-2\lambda & \lambda & \multicolumn{2}{c}{\ddots} &
      % & \multirow{2}{*}{\vdots} \\
      % 0 & & \multirow{2}{*}{\ddots} \\
      % \vdots & & & & & & 0 \\
      % 0 & \multicolumn{2}{c}{\cdots} & 0 & \lambda & 1-2\lambda \\
      1+2\lambda & -\lambda    & 0       & \multicolumn{2}{c}{\cdots} & 0 \\
      -\lambda    & 1+2\lambda & -\lambda & \ddots &  & \multirow{2}{*}{\vdots} \\
      0 & \ddots & \ddots & \ddots & \ddots & & \\
      \multirow{2}{*}{\vdots} & \ddots & \ddots & \ddots & \ddots & 0 \\
      & & \ddots & -\lambda & 1+2\lambda & -\lambda \\
      0 & \multicolumn{2}{c}{\cdots} & 0 & -\lambda & 1+2\lambda \\
    \end{pmatrix} \\
    \\
    \mathbf{w}^{(0)} &= \sin\frac{\pi}{4}x \cdot
    \left(1+2\cos\frac{\pi}{4}x\right)\\
  \end{aligned}
\end{equation}

The discrete equation for the Backward-Difference Method is
\begin{equation}
  \label{eq:8b_discrete}
  \mathbf{w}_{i,j-1}=(1+2\lambda)w_{i,j}-\lambda(w_{i+1,j}+w_{i-1,j})
\end{equation}
Which turns out to be 
\begin{equation}
  \label{eq:8b_discrete_val}
  \mathbf{w}_{i,j-1}=\left(1+\frac{16}{5\pi^2}\right)w_{i,j}-\frac{8}{5\pi^2}(w_{i+1,j}+w_{i-1,j})
\end{equation}
where 
\begin{equation}
  \label{eq:8b_w_i_0}
  \mathbf{w}_{i,0} = \cos\left[\pi\left(x_i-\frac{1}{2}\right)\right]
\end{equation}
And now further approximations of $\mathbf{w}$ can be computed by
\begin{equation}
  \label{eq:8b_wnext}
  \mathbf{w}^{(j-1)} = A\mathbf{w}^{(j)}
\end{equation}

In the Backward-Difference method, higher orders of $\mathbf{w}$ can
only be obtained by solving the linear system for $\mathbf{w}^{(j)}$
in \cref{eq:8b_wnext}.

Using the algorithm given in \cref{lst:backdiff}, I obtained the data
in \cref{tab:8b_1,tab:8b_2}.

\begin{table}[H]
  \centering
  \begin{tabular}[p]{cccc}
    \hline
    $x_i$ & $u(x_i,1/10)$ & $w(i,1/10)$ & $|u(x_i,1/10)-w(i,1/10)|$ \\
    \hline
    0 & 0 & 0 & \\
    0.25 & 0.63982 & 0.64578 & 0.0059641 \\
    0.50 & 0.90484 & 0.91327 & 0.0084345 \\
    0.75 & 0.63982 & 0.64578 & 0.0059641 \\
    1.00 & 0 & 0 & \\
    \hline
  \end{tabular}
  \caption{Data for Number 8b for t=1/10}
  \label{tab:8b_1}
\end{table}

\begin{table}[H]
  \centering
  \begin{tabular}[p]{cccc}
    \hline
    $x_i$ & $u(x_i,2/10)$ & $w(i,2/10)$ & $|u(x_i,2/10)-w(i,2/10)|$ \\
    \hline
    0 & 0 & 0 & \\
    0.25 & 0.57893 & 0.58977 & 0.010843 \\
    0.50 & 0.81873 & 0.83407 & 0.015335 \\
    0.75 & 0.57893 & 0.58977 & 0.010843 \\
    1.00 & 0 & 0 & \\
    \hline
  \end{tabular}
  \caption{Data for Number 8b for t=2/10}
  \label{tab:8b_2}
\end{table}


\subsection{10$^1$}
Repeat the question in \cref{sec:6} using the Crank-Nicolson Algorithm.
\subsubsection{b}

\Cref{eq:6b_const} still applies to this problem. We can construct the
$A$ and $B$ matrices and initial $\mathbf{w}$ vector using $\lambda$ and $u(x,0)$.
\begin{equation}
  \label{eq:10b_A}
  \begin{aligned}
    A &=
    \begin{pmatrix}
      1+\lambda & -\frac{\lambda}{2} & 0 & \multicolumn{2}{c}{\cdots} & 0 \\
      -\frac{\lambda}{2} & 1+\lambda & -\frac{\lambda}{2} & \ddots &  & \multirow{2}{*}{\vdots} \\
      0 & \ddots & \ddots & \ddots & \ddots & & \\
      \multirow{2}{*}{\vdots} & \ddots & \ddots & \ddots & \ddots & 0 \\
      & & \ddots & -\frac{\lambda}{2} & 1+\lambda & -\frac{\lambda}{2} \\
      0 & \multicolumn{2}{c}{\cdots} & 0 & -\frac{\lambda}{2} & 1+\lambda \\
    \end{pmatrix} \\
    B &=
    \begin{pmatrix}
      1-\lambda & \frac{\lambda}{2} & 0 & \multicolumn{2}{c}{\cdots} & 0 \\
      \frac{\lambda}{2} & 1-\lambda & \frac{\lambda}{2} & \ddots &  & \multirow{2}{*}{\vdots} \\
      0 & \ddots & \ddots & \ddots & \ddots & & \\
      \multirow{2}{*}{\vdots} & \ddots & \ddots & \ddots & \ddots & 0 \\
      & & \ddots & \frac{\lambda}{2} & 1-\lambda & \frac{\lambda}{2} \\
      0 & \multicolumn{2}{c}{\cdots} & 0 & \frac{\lambda}{2} & 1-\lambda \\
    \end{pmatrix} \\
    \\
    \mathbf{w}^{(0)} &= \sin\frac{\pi}{4}x \cdot
    \left(1+2\cos\frac{\pi}{4}x\right)\\
  \end{aligned}
\end{equation}

The discrete equation for the Crank-Nicolson method is
\begin{equation}
  \label{eq:10b_discrete}
  \frac{w_{i,j+1}-w_{i,j}}{k}=\frac{\alpha^2}{2}
  \left[\frac{w_{i+1,j}-2w_{i,j}+w_{i-1,j}}{h^2}
    + \frac{w_{i+1,j}-2w_{i,j}+w_{i-1,j}}{h^2}\right]
\end{equation}
Which simplifies to
\begin{equation}
  \label{eq:10b_discrete_simp}
  w_{i,j+1}-w_{i,j}=\frac{\lambda}{2}
  [w_{i+1,j}-2w_{i,j}+w_{i-1,j} + w_{i+1,j}-2w_{i,j}+w_{i-1,j}]
\end{equation}
And turns out to look like
\begin{equation}
  \label{eq:10b_discrete_simp}
  w_{i,j+1}-w_{i,j}=\frac{4}{5\pi^2}
  [w_{i+1,j}-2w_{i,j}+w_{i-1,j} + w_{i+1,j}-2w_{i,j}+w_{i-1,j}]
\end{equation}
where 
\begin{equation}
  \label{eq:10b_w_i_0}
  \mathbf{w}_{i,0} = \cos\left[\pi\left(x_i-\frac{1}{2}\right)\right]
\end{equation}
And now further approximations of $\mathbf{w}$ can be computed by
\begin{equation}
  \label{eq:10b_wnext}
  A\mathbf{w}^{(j+1)} = B\mathbf{w}^{(j)}
\end{equation}

In the Crank-Nicolson method, higher orders of $\mathbf{w}$ can
only be obtained by solving the linear system for
$\mathbf{w}^{(j+1)}$ in \cref{eq:10b_wnext}.


Using the algorithm given in \cref{lst:backdiff}, I obtained the data
in \cref{tab:10b_1,tab:10b_2}.

\begin{table}[H]
  \centering
  \begin{tabular}[p]{cccc}
    \hline
    $x_i$ & $u(x_i,1/10)$ & $w(i,1/10)$ & $|u(x_i,1/10)-w(i,1/10)|$ \\
    \hline
    0 & 0 & 0 & \\
    0.25 & 0.63982 & 0.64300 & 0.0031842 \\
    0.50 & 0.90484 & 0.90934 & 0.0045032 \\
    0.75 & 0.63982 & 0.64300 & 0.0031842 \\
    1.00 & 0 & 0 & \\
    \hline
  \end{tabular}
  \caption{Data for Number 10b for t=1/10}
  \label{tab:10b_1}
\end{table}

\begin{table}[H]
  \centering
  \begin{tabular}[p]{cccc}
    \hline
    $x_i$ & $u(x_i,2/10)$ & $w(i,2/10)$ & $|u(x_i,2/10)-w(i,2/10)|$ \\
    \hline
    0 & 0 & 0 & \\
    0.25 & 0.57893 & 0.58471 & 0.0057767 \\
    0.50 & 0.81873 & 0.82690 & 0.0081695 \\
    0.75 & 0.57893 & 0.58471 & 0.0057767 \\
    1.00 & 0 & 0 & \\
    \hline
  \end{tabular}
  \caption{Data for Number 10b for t=2/10}
  \label{tab:10b_2}
\end{table}

\section{Programming Minilab}
\subsection{b}
\begin{table}[H]
  \centering
  \begin{tabular}[H]{ccccc}
    \hline
    $i$ & $u_i(0.6, 0.6, 5)$ & $\Delta x_i$ & $\Delta y_i$ & $\Delta t_i$ \\
    \hline
    1 & 0.95370 & $\frac{1}{10}$ & $\frac{1}{10}$ & $\frac{1}{20}$ \\ 
    2 & 0.95438 & $\frac{1}{15}$ & $\frac{1}{15}$ & $\frac{1}{25}$ \\
    3 & -1.76134 $\times 10^{34}$ & $\frac{1}{20}$ & $\frac{1}{20}$ &
    $\frac{1}{30}$ \\
    \hline
  \end{tabular}
  \caption{Data for Minilab Part c}
  \label{tab:mini_b}
\end{table}

It seems that the computation does not approach a definite value
because it is unstable at higher resolutions of $t$. That is, the
value of $\Delta t$ at steps 2 and 3 is above what can be considered
stable for the Forward Difference Method, i.e. $\Delta t = 0.05 \not\leq
\frac{\Delta x^2\Delta y^2}{2P\Delta x^2 + 2Q\Delta y^2} < 0.083333$
for step 2 and $\Delta t = 0.04 \not\leq \frac{\Delta x^2\Delta y^2}{2P\Delta
  x^2 + 2Q\Delta y^2}<0.020833$ at step 3. We are not guaranteed
stability at steps 2 and 3, so the solution does not converge to a
definite value.
\subsection{c}
\begin{table}[H]
  \centering
  \begin{tabular}[H]{ccc}
    \hline 
    $i$ & $t_i$ & $u_{avg}(t_i)$ \\
    \hline
    1 & 0.5 & 0.9917993860561917 \\
    2 & 3   & 0.9640316857440158 \\
    3 & 10  & 0.9357088345473462 \\
    4 & 20  & 0.9248950884495316 \\
    5 & 40  & 0.9221110405827277 \\
    \hline
  \end{tabular}
  \caption{Data for Minilab Part c}
  \label{tab:mini_c}
\end{table}
\section{Code}
\lstinputlisting
[language=Octave,label=lst:forwarddiff,caption=\texttt{forwarddiff.m}]
{forwarddiff.m}
\pagebreak
\lstinputlisting
[language=Octave,label=lst:backdiff,caption=\texttt{backdiff.m}]
{backdiff.m}
\pagebreak
\lstinputlisting
[language=C++,label=lst:p12,caption=\texttt{program12.cpp}]
{program12.cpp}
\pagebreak
\lstinputlisting
[language=C++,label=lst:fwddiff2D,caption=\texttt{fwddiff2D.cpp}]
{fwddiff2D.cpp}

\end{document}