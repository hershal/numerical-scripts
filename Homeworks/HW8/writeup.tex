\documentclass[12pt]{article}
\title{M368K Homework 8 \\ \normalsize{\S~10.3 \#2b$^1$,10b\quad \S~10.4 \#1b$^2$\quad \S~10.5 \#1b$^3$,2c$^3$}}
\author{Hershal Bhave (hb6279)}
\date{\today}

\usepackage{listings}
\usepackage{mathtools}
\usepackage[in]{fullpage}
\usepackage{color}
\usepackage{tabularx}
\usepackage{courier}
\usepackage{caption}
\usepackage{cleveref}
\usepackage{nameref}
\usepackage{titlesec}
\usepackage{multicol}

% \relpenalty=9999
% \binoppenalty=9999

\definecolor{mygreen}{rgb}{0,0.6,0}
% \definecolor{mygreen}{rgb}{0.13,0.55,0.13}
\definecolor{mygray}{rgb}{0.5,0.5,0.5}
\definecolor{mymauve}{rgb}{0.58,0,0.82}

\lstset{
  backgroundcolor=\color{white},
  basicstyle=\footnotesize\ttfamily,
  breakatwhitespace=false,
  breaklines=true,
  captionpos=b,
  commentstyle=\color{mygreen},
  deletekeywords={...},
  escapeinside={\%*}{*)},
  extendedchars=true,
  frame=single,
  keywordstyle=\color{blue},
  % language=Octave,
  % numbers=left,
  % numbersep=5pt,
  % numberstyle=\tiny\color{mygray},
  rulecolor=\color{black},
  showspaces=false,
  showstringspaces=false,
  showtabs=false,
  % stepnumber=2,
  stringstyle=\color{mymauve},
  tabsize=2,
  title=\lstname,
  columns=fullflexible,
}

\begin{document}
\maketitle

\section{\S~10.3}
\subsection{2b$^{1}$}
Use Broyden's method with $\mathbf{x^{(0)}}$ = $\mathbf{0}$ to compute
$\mathbf{x^{(2)}}$ for the following nonlinear system, given
$\mathbf{x^{(0)}}$=
$\begin{pmatrix}
  5\\
  2\\
  0\\
\end{pmatrix}$

\begin{align*}
  \label{eq:2b}
  x^2_1+x_2-37&=0,\\
  x_1-x^2_2-5&=0,\\
  x_1+x_2+x_3-3&=0.\\
\end{align*}

Using the algorithm specified
in~\texttt{\nameref{lst:broyden}}~(\cref{lst:broyden}), I achieved the
following:\\

\begin{centering}
  \begin{tabularx}{\textwidth}{*5{>{\centering\arraybackslash}X}}
    \hline
    $k$ & $x^{(k)}_1$  & $x^{(k)}_2$ & $x^{(k)}_3$ & $\|\mathbf{x}^{(k)}-\mathbf{x}^{(k-1)}\|_2$ \\
    \hline
    0 & 5 & 2 & 0 &  \\
    1 & 6.0732 & 1.2683 & -4.3415 & 4.5316 \\
    2 & 5.9790 & 1.1178 & -4.0967 & 0.30241 \\
    3 & 6.0040 & 1.0413 & -4.0453 & 0.095495 \\
    $\vdots$ &$\vdots$ & $\vdots$ & $\vdots$ & $\vdots$ \\
    $\infty$ & 6 & 1 & -4 & $\lim_{k\rightarrow\infty}=0$ \\
    \hline
  \end{tabularx}
\end{centering}

\begin{minipage}{1.0\linewidth}
  \lstinputlisting[language=Octave,label=lst:broyden,caption=\texttt{broyden.m}]{broyden.m}
\end{minipage}

\newpage
\subsection{10b}
By multiplying on the right by $A + \mathbf{x y^{t}}$, show that when
$\mathbf{y^{t}} A^{-1} \mathbf{x} \ne -1$ we have
\begin{equation}
  \label{eq:10b}
  (A+\mathbf{x y^{t}})^{-1} = A^{-1} - \frac{A^{-1} \mathbf{x y^{t}} A^{-1}}{1+\mathbf{y^{t}}A^{-1}\mathbf{x}}
\end{equation}

\section{\S~10.4}
\subsection{1b$^2$}
Use the method of Steepest Descent to approximate $\mathbf{x}^{(1)}$
given $\mathbf{x}^{(0)}=(1,1.5)$ and $g(x)=\|F(x)\|_2^2$ and
$\alpha_0=1$ for the following nonlinear system:
\begin{align*}
  3x_1^2-x_2^2&=0, \\
  3x_1x_2^2-x_1^2-1&=0. \\
\end{align*}

\section{\S~10.5}
\subsection{1b$^3$}
The nonlinear system
\begin{align*}
  f_1(x_1,x_2)&=x_1^2-x_2^2+2x_2 =0 \\
  f_2(x_1,x_2)&=2x_1+x_2^2-6 =0 \\
\end{align*}
has two solutions,

\begin{equation*}
  \begin{array}{ccc}
    \mathbf{x}^{(1)}=
    \begin{bmatrix}
      0.625204094 \\
      2.179355825 \\
    \end{bmatrix}
    &
    \mathrm{and}
    &
    \mathbf{x}^{(2)}=
    \begin{bmatrix}
      2.109511920 \\
      −1.334532188 \\
    \end{bmatrix}
  \end{array}
\end{equation*}

Use the continuation method and Euler's method with $N=2$ to
approximate the solutions where $$\mathbf{x}(0)=(1,1)^t$$ and identify
which of the two known solutions the continuation curve is
approaching.

\subsection{2c$^3$}
The nonlinear system
\begin{align*}
  f_1(x_1,x_2)&=x_1^2-x_2^2+2x_2 =0 \\
  f_2(x_1,x_2)&=2x_1+x_2^2-6 =0 \\
\end{align*}
has two solutions,

\begin{equation*}
  \begin{array}{ccc}
    \mathbf{x}^{(1)}=
    \begin{bmatrix}
      0.625204094 \\
      2.179355825 \\
    \end{bmatrix}
    &
    \mathrm{and}
    &
    \mathbf{x}^{(2)}=
    \begin{bmatrix}
      2.109511920 \\
      −1.334532188 \\
    \end{bmatrix}
  \end{array}
\end{equation*}
Use the Runge-Kutta method of order four with order four with $N=1$ to
approximate the solutions where $$\mathbf{x}(0)=(1,1)^t$$ and identify
which of the two known solutions the continuation curve is
approaching.


\end{document}
