\documentclass[12pt]{article}
\title{M368K Homework 8 \\ \normalsize{\S~10.3 \#2b$^1$,10b\quad \S~10.4 \#1b$^2$\quad \S~10.5 \#1b$^3$,2c$^3$}}
\author{Hershal Bhave (hb6279)}
\date{\today}

\usepackage{listings}
\usepackage{mathtools}
\usepackage[in]{fullpage}
\usepackage{color}
\usepackage{tabularx}
% \usepackage{courier}
\usepackage{caption}
\usepackage{cleveref}
\usepackage{nameref}
\usepackage{titlesec}
\usepackage{framed}

% \relpenalty=9999
% \binoppenalty=9999

\definecolor{mygreen}{rgb}{0,0.6,0}
% \definecolor{mygreen}{rgb}{0.13,0.55,0.13}
\definecolor{mygray}{rgb}{0.5,0.5,0.5}
\definecolor{mymauve}{rgb}{0.58,0,0.82}

\lstset{
  backgroundcolor=\color{white},
  basicstyle=\scriptsize\ttfamily,
  breakatwhitespace=false,
  breaklines=true,
  captionpos=b,
  commentstyle=\color{mygreen},
  deletekeywords={...},
  escapeinside={\%*}{*)},
  extendedchars=true,
  frame=single,
  keywordstyle=\color{blue},
  % language=Octave,
  % numbers=left,
  % numbersep=5pt,
  % numberstyle=\tiny\color{mygray},
  rulecolor=\color{black},
  showspaces=false,
  showstringspaces=false,
  showtabs=false,
  % stepnumber=2,
  stringstyle=\color{mymauve},
  tabsize=2,
  title=\lstname,
  columns=fullflexible,
}

\begin{document}
\maketitle

\section{\S~10.3}
\subsection{2b$^{1}$}
Use Broyden's method with $\mathbf{x^{(0)}}$ = $\mathbf{0}$ to compute
$\mathbf{x^{(2)}}$ for the following nonlinear system, given
$\mathbf{x^{(0)}}$=
$\begin{pmatrix}
  5\\
  2\\
  0\\
\end{pmatrix}$

\begin{align*}
  \label{eq:2b}
  x^2_1+x_2-37&=0,\\
  x_1-x^2_2-5&=0,\\
  x_1+x_2+x_3-3&=0.\\
\end{align*}

Using the algorithm specified
in~\texttt{\nameref{lst:broyden}} (\cref{lst:broyden}), I achieved the
following:\\

\begin{centering}
\boxed{
  \begin{tabularx}{\textwidth}{*5{>{\centering\arraybackslash}X}}
    \hline
    $k$ & $x^{(k)}_1$  & $x^{(k)}_2$ & $x^{(k)}_3$ & $\|\mathbf{x}^{(k)}-\mathbf{x}^{(k-1)}\|_2$ \\
    \hline
    0 & 5 & 2 & 0 &  \\
    1 & 6.0732 & 1.2683 & -4.3415 & 4.5316 \\
    2 & 5.9790 & 1.1178 & -4.0967 & 0.30241 \\
    3 & 6.0040 & 1.0413 & -4.0453 & 0.095495 \\
    $\vdots$ &$\vdots$ & $\vdots$ & $\vdots$ & $\vdots$ \\
    $\infty$ & 6 & 1 & -4 & $\lim_{k\rightarrow\infty}=0$ \\
    \hline
  \end{tabularx}
  }
\end{centering}

\begin{minipage}{1.0\linewidth}
  \lstinputlisting[language=Octave,label=lst:broyden,caption=\texttt{broyden.m}]{broyden.m}
\end{minipage}

\newpage
\subsection{10b}
By multiplying on the right by $A + \mathbf{x y^{t}}$, show that when
$\mathbf{y^{t}} A^{-1} \mathbf{x} \ne -1$ we have
\begin{equation}
  \label{eq:10b}
  (A+\mathbf{x y^{t}})^{-1} = A^{-1} - \frac{A^{-1} \mathbf{x y^{t}} A^{-1}}{1+\mathbf{y^{t}}A^{-1}\mathbf{x}}
\end{equation}

By right-multiplying the LHS with $A + \mathbf{x y^{t}}$, we obtain 

\begin{align*}
I &= (A+\mathbf{x y^t}) A^{-1} - \frac{A^{-1} \mathbf{x} \mathbf{y}^{t} A^{-1}}{1+\mathbf{y}^{t}A^{-1}\mathbf{x}}\\
&= AA^{-1}+\mathbf{x}\mathbf{y}^t A^{-1}-\frac{A^{-1}A \mathbf{x} \mathbf{y}^{t} A^{-1}+\mathbf{x} \mathbf{y}^{t}A^{-1}\mathbf{x} \mathbf{y}^{t}A^{-1}}{1+\mathbf{y}^{t}A^{-1}\mathbf{x}}\\
&=I+\mathbf{x y^t} A^{-1}-\frac{\mathbf{x}(1+\mathbf{y^{t}}A^{-1}\mathbf{x})\mathbf{y^{t}}}{1+\mathbf{y^{t}}A^{-1}\mathbf{x}}\\
&=I+\mathbf{x} \mathbf{y}^{t}A^{-1}-\mathbf{x} \mathbf{y}^{t}A^{-1}\\
I&= I.\\
\end{align*}
The equality seems to hold, so we have shown that when $\mathbf{y^{t}} A^{-1} \mathbf{x} \ne -1$, Equation~\ref{eq:10b} holds true.\\
\section{\S~10.4}
\subsection{1b$^2$}
Use the method of Steepest Descent to approximate $\mathbf{x}^{(1)}$
given $\mathbf{x}^{(0)}=(1,1.5)$, $g(x)=\|F(x)\|_2^2$ and $\alpha_0=1$
for the following nonlinear system:
\begin{equation*}
  \label{eq:stp_F}
  F(\mathbf{x}) = \left\{
    \begin{aligned}
      3x_1^2-x_2^2&=0, \\
      3x_1x_2^2-x_1^2-1&=0. \\
    \end{aligned}
  \right.
\end{equation*}
$g(\mathbf{x})$ is given as follows:
\begin{equation}
  \label{eq:stp_g}
  g(\mathbf{x}) = f_1(x_1,x_2)^2+f_2(x_1,x_2)^2
\end{equation}

Which implies its gradient is:
\begin{equation}
  \label{eq:stp_nabg}
  \nabla g(x_1,x_2) \equiv \nabla g(\mathbf{x}) = 
  \begin{pmatrix}
    2f_1(\mathbf{x})\frac{\partial f_1}{\partial x_1}(\mathbf{x}) + 2f_2(\mathbf{x})\frac{\partial f_2}{\partial x_1}(\mathbf{x}) \\
    2f_1(\mathbf{x})\frac{\partial f_1}{\partial x_2}(\mathbf{x}) + 2f_2(\mathbf{x})\frac{\partial f_2}{\partial x_2}(\mathbf{x}) \\
  \end{pmatrix}
\end{equation}

For $\mathbf{x}^{(0)}=(1,1.5)$, we have 
\begin{equation*}
  g(\mathbf{x}^{(0)}) = 14.625 \text{, } \quad
  \nabla g(\mathbf{x^{(0)}}) =
  \begin{pmatrix}
    44.625 \\
    63.000 \\
  \end{pmatrix} \text{, and } \quad
  z_0=\|\nabla g(\mathbf{x^{(0)}})\|_2 = 77.204
.
\end{equation*}

Now we will find a normalized $\mathbf{z}$. Let
\begin{equation*}
  \mathbf{z} = \frac{1}{z_0}\nabla g(\textbf{x}^{(0)}) = 
  \begin{pmatrix}
    0.57802 \\
    0.81602 \\
  \end{pmatrix}
\end{equation*}

% With $\alpha_1=0$, we have 
% $$g_1=g(\mathbf{x}^{(0)}-\alpha_1\mathbf{z}) = g(\mathbf{x}^{(0)}) = 14.625$$.
% We can arbitrarily let $\alpha_3=1$ so that
% $$g_3=g(\mathbf{x}^{(0)}-\alpha_3\mathbf{z}) = 1.57123$$.
% Because $g_3<g_1$ , we accept $\alpha_3$ and set $\alpha_2$ = $\alpha_3/2 = 0.5$. Thus
% $$g_2=g(\mathbf{x}^{(0)}-\alpha_2\mathbf{z}) = 0.11978$$.
% 
% We now find the quadratic polynomial that interpolates the data $(0, 14.625)$,
% $(1, 1.57123)$, and $(0.5, 0.11978)$. It is most convenient to use Newton’s
% foward divided-difference inter polating polynomial for this purpose, which has
% the form
% \begin{equation}
%   \label{eq:stp_P}
%   P(\alpha) = g_1+h_1\alpha+h_3\alpha(\alpha-\alpha_2)
% \end{equation}
% 
% Which interpolates
% $$g(\textbf{x}^{(0)}-\alpha\nabla g(\textbf{x}^{(0)}))=g(\textbf{x}^{(0)}-\alpha\textbf{z})$$
% 
% at $\alpha_1=0$, $\alpha_2=0.5$, and $\alpha_3=1$ as follows:
% 
% \begin{center}
% $
%   \begin{array}{llll}
%     \alpha_1=0, & g_1=14.625, & \\
%     \alpha_2=0.5, & g_2=0.11978, & h_1=-29.010 \\
%     \alpha_3=1, & g_3=1.57123, & h_2=2.9029, & h_3=31.913 \\
%   \end{array}
% $
% \end{center}
% 
% Thus 
% $$P(\alpha)=14.625-(-29.010)\alpha+31.913\alpha(\alpha-\alpha_2)$$
% is our interpolating polynomial given by the form in Equation~\ref{eq:stp_P}.

We are given that $\alpha_0=0$ so we will skip the generation of an
interpolating polynomial. All we have to do now is simply
\begin{equation}
  \label{eq:stp_x1}
  \mathbf{x}^{(1)}=\mathbf{x}^{(0)}-\alpha_0\mathbf{z}
\end{equation}

Using Equation~\ref{eq:stp_x1} we obtain
\begin{equation*}
  \label{eq:stp_x1_numeric}
  \mathbf{x}^{(1)}=
  \begin{pmatrix}
    1 \\
    1.5 \\
  \end{pmatrix} - 
  \begin{pmatrix}
    0.57802 \\
    0.81602 \\
  \end{pmatrix}  
\end{equation*}
Which gives us our final answer
\begin{equation*}
\label{eq:stp_x1_answer}
\boxed{\mathbf{x}^{(1)}=
  \begin{pmatrix}
    0.42198 \\
    0.68398 \\
  \end{pmatrix}}
\end{equation*}

\begin{minipage}[h]{1.0\linewidth}
  \lstinputlisting[language=Octave,label=lst:steepest,caption=\texttt{steepest.m}]{steepest.m}
\end{minipage}

\section{\S~10.5}
\subsection{1b$^3$}
The nonlinear system
\begin{align*}
  f_1(x_1,x_2)&=x_1^2-x_2^2+2x_2 =0 \\
  f_2(x_1,x_2)&=2x_1+x_2^2-6 =0 \\
\end{align*}
has two solutions,

\begin{equation*}
  \begin{array}{ccc}
    \mathbf{x}^{(1)}=
    \begin{bmatrix}
      0.625204094 \\
      2.179355825 \\
    \end{bmatrix}
    &
    \mathrm{and}
    &
    \mathbf{x}^{(2)}=
    \begin{bmatrix}
      2.109511920 \\
      -1.334532188 \\
    \end{bmatrix}
  \end{array}
\end{equation*}

Use the continuation method and Euler's method with $N=2$ to approximate the
solutions where $\mathbf{x}(0)=(1,1)^t$ and identify which of the two known
solutions the continuation curve is approaching.

Using the algorithm for Euler's Method of order four and $N=2$ described in
in~\texttt{\nameref{lst:eulers}} (\cref{lst:eulers}), I was able to obtain the following:

\begin{equation*}
  \boxed{
    \begin{array}[h]{cc}
    \mathbf{x}^*= \left\{
      \begin{array}{c}
        0.42105 \\
        2.61842 \\
      \end{array}
    \right.\\
    \text{The solution appears to be approaching }\mathbf{x}^{(1)}\text{.}
  \end{array}
  }
\end{equation*}
\begin{minipage}[h]{1.0\linewidth}
  \lstinputlisting[language=Octave,label=lst:eulers,caption=\texttt{eulers.m}]{eulers.m}
\end{minipage}

\subsection{2c$^3$}
The nonlinear system
\begin{align*}
  f_1(x_1,x_2)&=x_1^2-x_2^2+2x_2 =0 \\
  f_2(x_1,x_2)&=2x_1+x_2^2-6 =0 \\
\end{align*}
has two solutions,

\begin{equation*}
  \begin{array}{ccc}
    \mathbf{x}^{(1)}=
    \begin{bmatrix}
      0.625204094 \\
      2.179355825 \\
    \end{bmatrix}
    &
    \mathrm{and}
    &
    \mathbf{x}^{(2)}=
    \begin{bmatrix}
      2.109511920 \\
      -1.334532188 \\
    \end{bmatrix}
  \end{array}
\end{equation*}
Use the Runge-Kutta method of order four with order four with $N=1$ to
approximate the solutions where $\mathbf{x}(0)=(3,-2)^t$ and identify
which of the two known solutions the continuation curve is
approaching.

Using the algorithm for Runge-Kutta of order four and $N=1$ described in
in~\texttt{\nameref{lst:rungekutta}} (\cref{lst:rungekutta}), I was able to obtain the following:

\begin{equation*}
  \boxed{
    \begin{array}[h]{cc}
    \mathbf{x}^*= \left\{
      \begin{array}{c}
        2.1094 \\
        -1.3346 \\
      \end{array}
    \right.\\
    \text{The solution appears to be approaching }\mathbf{x}^{(2)}\text{.}
  \end{array}
  }
\end{equation*}
\begin{minipage}[h]{1.0\linewidth}
  \lstinputlisting[language=Octave,label=lst:rungekutta,caption=\texttt{rungekutta.m}]{rungekutta.m}
\end{minipage}

\section{Programming Minilab}
The solutions generally converge to the same point.

\end{document}
