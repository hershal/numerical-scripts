\documentclass[12pt]{article}
\title{M368K Homework 11 \\
  \normalsize{\S~11.5 \#12$^1$\quad \S~12.1 \#2, 8$^2$}}
\author{Hershal Bhave (hb6279)}
\date{Due April 19, 2013}

\usepackage{mathtools}
\usepackage{amssymb}
\usepackage{amsfonts}
\usepackage{mathabx}
\usepackage{listings}
\usepackage[in]{fullpage}
\usepackage{color}
\usepackage{tabularx}
\usepackage{caption}
\usepackage{cleveref}
\usepackage{titlesec}
\usepackage{float}
\usepackage{caption}
\usepackage{subcaption}
\usepackage{wrapfig}
\usepackage{cancel}

\newcommand{\dd}[1]{\mathrm{d}{#1}}
\newcommand{\ddt}[1]{\frac{\dd{}}{\dd{#1}}}
\newcommand{\dddt}[1]{\frac{\dd{}^2}{\dd{#1}^2}}

\definecolor{mygreen}{rgb}{0,0.6,0}
\definecolor{mygray}{rgb}{0.5,0.5,0.5}
\definecolor{mymauve}{rgb}{0.58,0,0.82}

\lstset{
  backgroundcolor=\color{white},
  basicstyle=\scriptsize\ttfamily,
  breakatwhitespace=false,
  breaklines=true,
  captionpos=b,
  commentstyle=\color{mygreen},
  deletekeywords={...},
  escapeinside={\%*}{*)},
  extendedchars=true,
  frame=single,
  keywordstyle=\color{blue},
  % language=Octave,
  % numbers=left,
  % numbersep=5pt,
  % numberstyle=\tiny\color{mygray},
  rulecolor=\color{black},
  showspaces=false,
  showstringspaces=false,
  showtabs=false,
  % stepnumber=2,
  stringstyle=\color{mymauve},
  tabsize=2,
  title=\lstname,
  columns=fullflexible,
}

\relpenalty=10000
\binoppenalty=10000

\begin{document}
\maketitle

\section{\S~11.5}
\subsection{12$^1$}\
\subsubsection{12$^1$a}
Show that the matrix given by the piecewise linear basis function is
positive definite. Assume $p(x)>0$ and $q(x)\geq 0$. Show $c^\intercal
Ac = \int_a^bp[v']^2+qv^2\dd{x} > 0$ where
$v(x)=\sum_{i=1}^Nc_i\phi_i(x)$.

\hfill

\begin{equation}
  \label{eq:12_1_1}
  \begin{aligned}
    c^\intercal Ac &= \sum_{i=1}^N \left(\sum_{j=1}^N c_i A_{i,j}c_j\right) \\
    &=\sum_{i=1}^N \left(\sum_{j=1}^N c_i \left[\int_a^b p \phi'_i \phi'_j
        \dd{x} + \int_a^b q \phi_i \phi_j \dd{x}\right] c_j\right) \\
    &= \int_a^b p \phi'^2 + q \phi^2 \dd{x} \\
  \end{aligned}
\end{equation}

Since it was assumed that $p(x)>0$ and $q(x)\geq0$ and it was given
that $v(x)=\sum_{i=1}^Nc_i\phi_i(x)$, we know that
$v(x)^2\geq0$. Since $v(x)^2$ will always be greater than or equal to
zero, the integral $\int_a^bp[v']^2+qv^2\dd{x}$ will also be greater
than or equal to zero based on $v(x)^2\geq0$ and the previous
assumptions. This implies that $c^\intercal Ac$ will alway be greater
than or equal to zero for any $c \in \mathbb{R}^n$

\hfill $\blacksquare$

\subsubsection{12$^1$b}
Explain each of the implications $c^\intercal
Ac =0 \implies v'(x) \equiv 0$ in each subinterval $(x_i,x_{i+1})
\implies v(x) \equiv 0$ in $[a,b] \implies c=0$ and conclude that $A$
is positive definite.

\hfill

If $c^\intercal Ac = \int_a^bp[v']^2+qv^2\dd{x} = 0$ then both
$\int_a^bp[v']^2\dd{x}$ and $\int_a^bqv^2\dd{x}$ are zero. If we
assume that $v'(x)=0$ in each subinterval $(x_i,x_{i+1})$, then that
must imply that $v(x)=k$ for some $k \in \mathbb{R}$. Furthermore, if
$v(x)=k$ and we assume that $v(a) = v(b) = 0$, then we must conclude
that $v(x) = 0$. This implies that every weight $c_i$ in
$v(x)=\sum_{i=1}^Nc_i\phi_i(x)$ is zero, i.e. $c=0$.

Since $A=\int_a^bp[v']^2+qv^2\dd{x}$ is symmetric and $c^\intercal Ac$
is only zero if $c=0$ then $c^\intercal Ac>0$ for at least some
$c_i\neq0$. Thus $c^\intercal Ac$ is positive definite.

\hfill $\blacksquare$

\section{\S~12.1}
\subsection{2}
Use the Poisson Equation Finite Difference Algorithm to approximate
the solution to the elliptic partial difference equation.

\begin{equation*}
  \frac{\partial^2 u}{\partial x^2}+\frac{\partial^2 u}{\partial
    y^2}=0 \qquad 1<x<2,\quad 0<y<1 \\
\end{equation*}
\begin{equation*}
  \begin{aligned}
    u(x,0)&=2\ln x, &u(x,1)&=\ln(x^2+1),\quad &1\leq x\leq 2; \\
    u(1,y)&=\ln(y^2+1), &u(2,y)&=ln(y^2+4),\quad &0\leq y \leq 1. \\
  \end{aligned}
\end{equation*}
Use $h=k=\frac{1}{3}$ and compare the results to the actual solution
$u(x,y)=\ln(x^2+y^2)$.

\hfill

Using the algorithm defined in \cref{lst:poisson} with tolerance
$10^{-6}$, I obtained the data in \cref{tab:2}.
\begin{table}[h]
  \centering
%  \begin{tabularx}{.9\textwidth}{*7{>{\centering\arraybackslash}X}}
  \begin{tabular}{ccccccc}
      \hline
      i & j & $x_i$ & $y_i$ & $w_{i,j}^{(19)}$ & $u(x_i,y_i)$ & $\mid
      u(x_i,y_i) - w_{i,j}^{(19)}\mid$ \\
      \hline
      1 & 1 & 1.333 & 0.333 & 0.63480 & 0.63599 & $1.1844 \times 10^{-3}$ \\
      1 & 2 & 1.333 & 0.667 & 0.79850 & 0.79851 & $7.3735 \times 10^{-6}$ \\
      2 & 1 & 1.667 & 0.333 & 1.05999 & 1.06087 & $8.7950 \times 10^{-4}$ \\
      2 & 2 & 1.667 & 0.667 & 1.16982 & 1.17007 & $2.5035 \times 10^{-4}$ \\
      \hline
  \end{tabular}
  \caption{Poisson Equation Finite Difference Approximation for
    \S~12.1 Number 2}
  \label{tab:2}
\end{table}

\subsection{8$^2$}
A 6-cm by 5-cm rectangular silver plate has heat being uniformly
generated at each point at the rate
$q$. Let $x$ represent the
distance along the edge of the plate of length 6 cm and y be the
distance along the edge of the plate of length 5 cm. Suppose the
temperature $u$ along the edges is kept at the following temperatures:
\begin{equation*}
  \begin{aligned}
    u(x,0)&=x(6-x), &u(x,5)&=0,\quad &0\leq x\leq 6; \\
    u(0,y)&=y(5-y), &u(6,y)&=0,\quad &0\leq y \leq 5. \\
  \end{aligned}
\end{equation*}
where the origin lies at a corner of the plate with coordinates (0,0)
and the edges lie along the positive x- and y-axes. The steady-state
temperature $u=u(x,y)$ satisfies the Poisson's equation:
\begin{equation*}
  \frac{\partial^2u}{\partial x^2}(x,y) +
  \frac{\partial^2u}{\partial y^2}(x,y) = -\frac{q}{K}, \qquad
  0<x<6,\quad 0<y<5,
\end{equation*}
where $K$, the thermal conductivity, is $1.04 \text{
  cal}/\text{cm}\cdot \text{deg} \cdot s$ and $q$, the rate heat is
being generated, is $1.5\text{
  cal}/\text{cm}^3\cdot\text{s}$. Approximate the temperature $u(x,y)$
using the Poisson Equation Finite Difference Approximation Algorithm
in \cref{lst:poisson} and two interior nodes in each direction.

\hfill

The general centered-difference formulas are
\begin{equation}
  \label{eq:8_gencentdiff}
  \begin{aligned}
    \frac{\partial^2u}{\partial x^2}(x_i,y_j) &=
    \frac{u(x_{i+1},y_{j})-2u(x_{i},y_{j}) + u(x_{i-1},y_{j})}{h^2} -
    \frac{h^2}{12}\frac{\partial^4u}{\partial x^4}(\xi_i,y_j) \\
    \frac{\partial^2u}{\partial y^2}(x_i,y_j) &=
    \frac{u(x_{i},y_{j+1})-2u(x_{i},y_{j}) + u(x_{i},y_{j-1})}{k^2} -
    \frac{k^2}{12}\frac{\partial^4u}{\partial y^4}(x_i,\eta_j) \\
  \end{aligned}
\end{equation}
where $\xi_i \in (x_{i-1},x_{i+1})$ and $\eta_j \in
(y_{j-1},y_{j+1})$.
Writing this in difference-equation form, we get
\begin{equation}
  \label{eq:findiffcentdiff}
  2\left[\left(\frac{h}{k}\right)^2+1\right]w_{i,j}-(w_{i+1,j}+w_{i-1,j})
  - \left(\frac{h}{k}\right)^2(w_{i,j+1}+w_{i,j-1})=-h^2f(x_i,y_j)
\end{equation}
Which can simplify to 
\begin{equation}
  \label{eq:findiffcentdiffsimp}
    \beta w_{i,j}-(w_{i+1,j}+w_{i-1,j})
  - \alpha (w_{i,j+1}+w_{i,j-1})=\gamma
\end{equation}
Where $\alpha=\left(\frac{h}{k}\right)^2$, $\beta=2(\alpha+1)$, and
$\gamma=-h^2\frac{q}{K}$. Expressing this in terms of the relabled
interior grid points $w_i=u(P_i)$, we get the equations at the points
$P_i$.

\begin{equation}
  \label{eq:8_p}
  \begin{aligned}
    P_1&: \beta w_1 - w_2 - \alpha w_3
    &=w_{0,2}+\cancel{\alpha w_{1,3}} + \gamma &= \frac{50}{9} + \gamma \\
    P_2&: \beta w_2 - w_1 - \alpha w_4 
    &=\cancel{w_{3,2}} + \cancel{\alpha w_{2,3}} + \gamma &= \gamma \\
    P_3&: \beta w_3 - w_4 - \alpha w_1  &=w_{0,1}+\alpha
    w_{1,0} + \gamma &= \frac{50}{9} + 8\alpha +  \gamma \\
    P_4&: \beta w_4 - w_3 - \alpha w_2
    &=\cancel{w_{3,1}}+\alpha w_{2,0} + \gamma &= 8 +\gamma \\
  \end{aligned}
\end{equation}
In matrix form
\begin{equation}
  \label{eq:8_matrix}
  \begin{array}{ccc}
  \begin{pmatrix}
    \beta & -1 & -\alpha & 0 \\
    -1 & \beta & 0 & -\alpha \\
    -\alpha & 0 & \beta & -1 \\
    0 & -\alpha & -1 & \beta \\
  \end{pmatrix}
  & 
  \begin{pmatrix}
    w_1 \\
    w_2 \\
    w_3 \\
    w_4 \\
  \end{pmatrix}
  &=
  \begin{pmatrix}
    \frac{50}{9} + \gamma \\
    \gamma \\
    \frac{50}{9} + 8\alpha + \gamma \\
    8 + \gamma
  \end{pmatrix}
\end{array}
\end{equation}
Using the algorithm defined in \cref{lst:poisson} with tolerance
$10^{-6}$, I obtained the data in \cref{tab:2}.
\begin{table}[H]
  \centering
%  \begin{tabularx}{.9\textwidth}{*7{>{\centering\arraybackslash}X}}
  \begin{tabular}{ccccc}
      \hline
      i & j & $x_i$ & $y_i$ & $w_{i,j}^{(19)}$ \\
      \hline
      1 & 1 & 2 & 1.667 & 7.5719 \\
      1 & 2 & 2 & 3.333 & 5.4064 \\
      2 & 1 & 4 & 1.667 & 6.3206 \\
      2 & 2 & 4 & 3.333 & 4.1552 \\
      \hline
  \end{tabular}
  \caption{Poisson Equation Finite Difference Approximation for
    \S~12.1 Number 8$^2$}
  \label{tab:8}
\end{table}

\section{Minilab}
\subsection{Part a}
Program output for S~12.1 Number 8$^2$ is in \cref{tab:mini_a} and is
consistent with the results I obtained before.
\begin{table}[h]
  \centering
  \begin{tabular}{cccc}
    \hline
    $x_i$ & $y_i$ & $u_{i,j}$ \\
    \hline
    0.00000 & 0.00000 & 0.00000 \\
    0.00000 & 1.66667 & 5.55556 \\
    0.00000 & 3.33333 & 5.55556 \\
    0.00000 & 5.00000 & 0.00000 \\
    2.00000 & 0.00000 & 8.00000 \\
    2.00000 & 1.66667 & 7.57186 \\
    2.00000 & 3.33333 & 5.40645 \\
    2.00000 & 5.00000 & 0.00000 \\
    4.00000 & 0.00000 & 8.00000 \\
    4.00000 & 1.66667 & 6.32061 \\
    4.00000 & 3.33333 & 4.15520 \\
    4.00000 & 5.00000 & 0.00000 \\
    6.00000 & 0.00000 & 0.00000 \\
    6.00000 & 1.66667 & 0.00000 \\
    6.00000 & 3.33333 & 0.00000 \\
    6.00000 & 5.00000 & 0.00000 \\
    \hline
  \end{tabular}
  \caption{Data for Minilab part a}
\label{tab:mini_a}
\end{table}

\section{Part b}
The concentration of pollutatnts at the school's location (0.8, 0.8)
is approximately 6.880\% (0.06880). The maximum concentration of
pollutants appears at (0.36667,0.23333) and is approximately 23.171\%
(0.23171). 
\section{Part c}
With the second factory at location (0.2,0.6), the concentration of
pollutants at the school is approximately 0.012\% (0.00012) and the
maximum concentration of pollutants is approximately 27.998\%
(0.27998) at location (0.10000, 0.23333).

With the second factory at location (0.8,0.2), the concentration of
pollutants at the school is approximately 0.09\% (0.00090) and the maximum
concentration of pollutants is approximately 35.709\% (0.35709) at
location (0.13333, 0.20000).

It would seem that having the second factory at location (0.8,0.2)
would have the lowest impact on the school under the conditions we assumed.
\pagebreak
\section{Code}
\subsection{Poisson Finite Difference Algorithm}
\lstinputlisting
[language=Octave,label=lst:poisson,caption=\texttt{poissonFinDiff.m}]
{../scripts/poissonFinDiff.m}
\pagebreak
\subsection{Program 11}
\lstinputlisting
[language=C++,label=lst:prog11,caption=\texttt{program11.cpp}]
{../src/program11.cpp}
\pagebreak
\subsection{Linear Centered-Difference 2D}
\lstinputlisting
[language=C++,label=lst:lincd,caption=\texttt{linearcd2D.cpp}]
{../src/linearcd2D.cpp}

\end{document}
