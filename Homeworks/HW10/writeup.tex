\documentclass[12pt]{article}
\title{M368K Homework 9 \\
  \normalsize{\S~11.1 \#2a$^1$, 9\quad \S~11.2 \#4a$^2$}}
\author{Hershal Bhave (hb6279)}
\date{April 12, 2013}

\usepackage{mathtools}
\usepackage{amssymb}
\usepackage{amsfonts}
\usepackage{mathabx}
\usepackage{listings}
\usepackage[in]{fullpage}
\usepackage{color}
\usepackage{tabularx}
\usepackage{caption}
\usepackage{cleveref}
\usepackage{titlesec}
\usepackage{float}
\usepackage{caption}
\usepackage{subcaption}
\usepackage{wrapfig}

\newcommand{\dd}[1]{\mathrm{d}{#1}}
\newcommand{\ddt}[1]{\frac{\dd{}}{\dd{#1}}}
\newcommand{\dddt}[1]{\frac{\dd{}^2}{\dd{#1}^2}}

\definecolor{mygreen}{rgb}{0,0.6,0}
% \definecolor{mygreen}{rgb}{0.13,0.55,0.13}
\definecolor{mygray}{rgb}{0.5,0.5,0.5}
\definecolor{mymauve}{rgb}{0.58,0,0.82}

\lstset{
  backgroundcolor=\color{white},
  basicstyle=\scriptsize\ttfamily,
  breakatwhitespace=false,
  breaklines=true,
  captionpos=b,
  commentstyle=\color{mygreen},
  deletekeywords={...},
  escapeinside={\%*}{*)},
  extendedchars=true,
  frame=single,
  keywordstyle=\color{blue},
  % language=Octave,
  % numbers=left,
  % numbersep=5pt,
  % numberstyle=\tiny\color{mygray},
  rulecolor=\color{black},
  showspaces=false,
  showstringspaces=false,
  showtabs=false,
  % stepnumber=2,
  stringstyle=\color{mymauve},
  tabsize=2,
  title=\lstname,
  columns=fullflexible,
}

\begin{document}
\maketitle

\section{\S~11.3}
\subsection{\S~2b$^1$}

The Boundary-value problem 
$$ y'' = y'+2y+\cos x, \qquad 0\leq x\leq\frac{\pi}{2},\qquad y(0)=-0.3,\qquad y\left(\frac{\pi}{2}\right)=-0.1 $$
has the solution $y(x) = -\frac{1}{10}(\sin x + 3\cos x)$. Use the
Linear Finite-Difference method to approximate the solution, and
explicitly write out the centered-difference equations. Solve and
compare the results to the actual solution. Assume $h=\frac{\pi}{4}$.

Given a differential equation in the form
\begin{equation}
  \label{eq:gendiff}
  y''(x_i)=p(x_i)y'(x_i)+q(x_i)y(x_i)+r(x_i),
\end{equation}

the third-order Taylor polynomial of $y(x_{i-1})$ and $y(x_{i-1})$ are
\begin{equation}
  \label{eq:taylorpoly}
  \begin{aligned}
    y(x_{i+1}) &= y(x_i+h) = y(x_i) + hy'(x_i)+\frac{h^2}{2}y''(x_i) + \frac{h^3}{3}y'''(x_i)+\frac{h^4}{24}y^{(4)}(\xi_i^+)+\ldots\\
    y(x_{i-1}) &= y(x_i-h) = y(x_i) - hy'(x_i)+\frac{h^2}{2}y''(x_i)
    - \frac{h^3}{3}y'''(x_i)+\frac{h^4}{24}y^{(4)}(\xi_i^+)-\ldots \\
  \end{aligned}
\end{equation}

Combining the equations in \cref{eq:taylorpoly} and solving for $y''$, we obtain the
Centered-Difference Formula for $y''$.
\begin{equation}
  \label{eq:centdiffypp}
  y''(x_i)=\frac{1}{h^2}[y(x_{i+1})-2y(x_i)+y(x_{i-1})]-\frac{h^2}{12}y^{(4)}(\xi_i) 
\end{equation}

A simliar Centeral Difference Equation can be found for $y'$.
\begin{equation}
  \label{eq:centdiffyp}
  y'(x_i)=\frac{1}{2h}[y(x_{i+1})-y(x_{i-1})]-\frac{h^2}{6}y'''(\eta_i)
\end{equation}

By ignoring the higher-order terms we can calculate an approximation
to the solution $y(x_i)$. Using the Centered-Difference Formula in
\cref{eq:centdiffypp} and the given equation for $y''(x_i)$ in the
form of \cref{eq:gendiff} we obtain
\begin{equation}
  \label{eq:findiff-prelim}
    \frac{y(x_{i+1})-2y(x_i)+y(x_{i-1})}{h^2} = p(x_i)y'(x_i)+q(x_i)y(x_i)+r(x_i)
\end{equation}
By applying \cref{eq:centdiffyp} to \cref{eq:findiff-prelim}, we obtain
\begin{equation}
  \label{eq:findiff}
  \frac{y(x_{i+1})-2y(x_i)+y(x_{i-1})}{h^2}=p(x_i)\frac{y(x_{i+1}) -
    y(x_{i-1})}{2h}+q(x_i)y(x_i) + r(x_i)
\end{equation}
which will allow us to write the approximations in matrix-form and
then solve the resulting matrix. Observe that we can simplify this
into
\begin{equation*}
  \label{eq:findiff-maty}
    -r(x_i) = \frac{y(x_{i+1})-2y(x_i)+y(x_{i-1})}{h^2} +
    p(x_i)\left(\frac{y(x_{i+1}-y(x_{i-1})}{2h}\right)+q(x_i)y(x_i)
\end{equation*}
or alternatively
\begin{equation}
  \label{eq:findiff-maty2}
  -h^2r(x_i) = \left(-1-\frac{h}{2}p(x_i)\right)y(x_{i-1}) +
  \left(2+h^2q(x_i)\right)y(x_i) - \left(-1+\frac{h}{2}p(x_i)\right)y(x_{i+1})
\end{equation}
Then applying \cref{eq:findiff-maty2} with $h=\frac{\pi}{4}$, we obtain
\begin{equation}
  \label{eq:findiff-sim}
  -\frac{\pi^2}{16}r(x_i) = \left(-1-\frac{\pi}{8}p(x_i)\right)y(x_{i-1}) +
  \left(2+\frac{\pi^2}{16}q(x_i)\right)y(x_i) - \left(-1+\frac{\pi}{8}p(x_i)\right)y(x_{i+1})
\end{equation}

By invoking our boundary value conditions and bounds we have $i=1$,
$x_0=0$, $x_1=\frac{\pi}{4}$, $x_2=\frac{\pi}{2}$, $y(0)=-0.3$, and
$y\left(\frac{\pi}{2}\right) = -0.1$. We can substitute the interior
values into \cref{eq:findiff-sim} to obtain an equation which, when
solved, describes $y(x_i)$ at the corresponding interior mesh points
of $x_i$. In our case the interior mesh point is $x_1$.
\begin{equation}
  \label{eq:findiff-values}
  \begin{aligned}
    -\frac{\pi^2\sqrt{2}}{32} &= \left(-1-\frac{\pi}{8}\right)y(x_0) +
    \left(2+\frac{\pi^2}{8}\right)y(x_1) -
    \left(-1+\frac{\pi}{8}\right)y(x_2) \\
    \left(2+\frac{\pi^2}{8}\right)y(x_1) &= -\frac{\pi^2\sqrt{2}}{32}
    + \left(-1-\frac{\pi}{2}\right)(0.3) -
    \left(-1+\frac{\pi}{2}\right)(0.1) \\
    &= \left[-\frac{\pi^2\sqrt{2}}{32} +
      \left(-1-\frac{\pi}{2}\right)(0.3) +
      \left(-1+\frac{\pi}{2}\right)(0.1)\right]\left(2+\frac{\pi^2}{8}\right)^{-1} \\
  \end{aligned}
\end{equation}
Finally we obtain our solution:
\begin{equation*}
  \boxed{
    y\left(\frac{\pi}{4}\right) = -0.28287
}
\end{equation*}

We can confirm this approximation is correct by running it through the
code in \cref{lst:linearfe} and approximating $y(x_i)$'s values at
more mesh points. Doing this with $n=3$ I obtained the data in
\cref{tab:2b} on \cpageref{tab:2b}, confirming that our solution is
close to the actual solution, even with only one step.

\begin{table}[hp]
  \centering
  \begin{tabularx}{\textwidth}{*4{>{\centering\arraybackslash}X}}
    \hline
    $x_i$ & $w_i$ & $y(x_i)$ & $|w_i-y(x_i)|$ \\
    \hline
    $0.00000$ & $-0.30000$ & $-0.30000$ & \\
    $0.39270$ & $-0.31569$ & $-0.31543$ & $2.5320 \times 10^{-4}$\\
    $0.78540$ & $-0.28291$ & $-0.28284$ & $6.3136 \times 10^{-5}$\\
    $1.17810$ & $-0.20700$ & $-0.20719$ & $1.9735 \times 10^{-4}$\\
    $1.57080$ & $-0.10000$ & $-0.10000$ & \\
    \hline
  \end{tabularx}
  \caption{Approximation of 2b with $n=3$}
  \label{tab:2b}
\end{table}
\subsection{9$^2$}
Use Theorem 9.1 to prove Theorem 11.3.

\begin{description}

\item[Theorem 9.1] Let A be an $n \times n$ matrix and $R_i$ denote
  the circle in the complex plane with center $a_{ii}$ and radius
  $\sum_{j=1,j\neq i}^{n}|a_{ij}|$; that is,
$$ R_i = \left\{ z \in \mathbb{C} \;\middle|\; |z-a_{ii}| \leq \sum_{j=1,j\neq
    i}^{n}|a_{ij}|\right\} $$
The eigenvalues of $A$ are containted within the union of these
circles, $R=\cup_{i=1}^{n}R_i$.
\item[Theorem 11.3] Suppose that $p$, $q$, and $r$ are continuous on
  $[a,b]$. If $q(x)\geq0$ on $[a,b]$, then the tridiagonal linear
  system has a unique solution provided that $h<\frac{2}{L}$, where
  $L=\text{max}_{a\leq x\leq b}|p(x)|$.
\end{description}
\textbf{Proof:}

By rearranging some of the inequality statements, Theorem 11.3 implies
$|\frac{h}{2}p(x)|<1$. Combining that knowledge with knowledge of the
tridiagonal matrix which defines the solution to the linear BVP, we
know that
\begin{align*}
  \sum_{j=1,j\neq i}^{n}|a_{ij}| &= |-1-\frac{h}{2}p(x_i)| +
  |-1-\frac{h}{2}p(x_i)| \\
  &= 1+\frac{h}{2}p(x_i) + 1 - \frac{h}{2}p(x_i) \\
\end{align*}
Which implies
$$ 0\leq \sum_{j=1,j\neq i}^{n}|a_{ij}| <2 $$

We know that the diagonal entries, $a_{ii}$ are composed of
$2+h^2q(x)$. Theorem 9.1 states that we must be able to find a radius
$z$ which satisfies
$$ R_i = \left\{ z \in \mathbb{C} \;\middle|\; |z-2-h^2q(x)| < 2\right\} $$
Since we also know that $q(x)$ is greater than zero, then this radius
must exist, meaning the matrix which defines the solutions to the BVP
must have precisely $k$ (counting multiplicities) of eigenvalues.
This means that the tridiagonal matrix is nonsingular, and thus has a
unique solution.

\hfill$ \blacksquare $

\section{\S~11.4}
\subsection{4a$^3$}
The Boundary-Value Problem
$$ y'' = y^3-yy',\qquad 1\leq x\leq2,\qquad y(1)=\frac{1}{2},\qquad y(2)=\frac{1}{3} $$
has the solution $y(x)=(x+1)^{-1}$. Use the Nonlinear
Finite-Difference Algorithm with $\text{TOL}=10^{-4}$ and $n=3$ to
approximate the solution. Explicity write out the centered-difference
equations and perform one Newton step with a straight guess of
$y^{(0)}$. Solve and compare the results to the actual solution.

Given a differential equation in the form 
\begin{equation}
  \label{eq:gendiffnonlin}
  y''(x_i)=f(x_i,y(x_i),y'(x_i))
\end{equation}
we can apply a similar method we used in \cref{eq:findiff} to find the
Centered-Difference equations in the nonlinear case.
\begin{equation}
  \label{eq:nonlinfindiff}
  \frac{y(x_{i+1})-2y(x_i)+y(x_{i-1})}{h^2} = f\left(x_i, y(x_i), \frac{y(x_{i+1})-y(x_{i-1})}{2h}\right)
\end{equation}
By invoking our boundary value conditions and bounds we have $i=1$,
$x_0=1$, $x_1=\frac{3}{2}$, $x_2=2$, $y(1)=\frac{1}{2}$, and
$y(2) = \frac{1}{3}$. Since $n=3$, $h=\frac{1}{4}$.
\begin{equation}
  \label{eq:nonlinfindiff-values}
  \frac{y(2)-2y(\frac{3}{2})+y(1)}{\frac{1}{16}} -
  f\left(\frac{3}{2}, y\left(\frac{3}{2}\right),
    \frac{y(2)-y(1)}{\frac{1}{2}}\right) = 0
\end{equation}
% The Jacobian associated with the Newton step is 
% 
% \begin{equation}
% \label{eq:jac}
% J=
% \begin{pmatrix}
% 2+\frac{1}{16}f_y\left(x_1,w_1,\frac{w_{2}-\alpha}{\frac{1}{2}}\right)
% &
% -1+\frac{1}{8}f_{y'}\left(x_1,w_1,\frac{w_{2}-\alpha}{\frac{1}{2}}\right)
% &
% 0
% \\
% 
% -1+\frac{1}{8}f_{y'}\left(x_2,w_2,\frac{w_{3}-w_{1}}{\frac{1}{2}}\right)
% &
% 2+\frac{1}{16}f_y\left(x_2,w_2,\frac{w_{3}-w_{1}}{\frac{1}{2}}\right)
% &
% -1+\frac{1}{8}f_{y'}\left(x_2,w_2,\frac{w_{3}-w_{1}}{\frac{1}{2}}\right)
% \\
% 
% 0
% &
% -1+\frac{1}{8}f_{y'}\left(x_3,w_3,\frac{\beta-w_{3}}{\frac{1}{2}}\right)
% &
% 2-\frac{1}{16}f_y\left(x_3,w_3,\frac{\beta-w_{3}}{\frac{1}{2}}\right)
% \end{pmatrix}
% \end{equation}

Performing one Newton Iteration, I obtained the table of values listed
in \cref{tab:4a}.

\begin{table}[H]
  \centering
  \boxed{
  \begin{tabularx}{\textwidth}{*4{>{\centering\arraybackslash}X}}
    \hline
    $x_i$ & $w_i$ & $y(x_i)$ & $|w_i-y(x_i)|$ \\
    \hline
    1.0000 & 0.50000 & 0.50000 &  \\
    1.2500 & 0.45833 & 0.44444 & 0.013889 \\
    1.5000 & 0.41667 & 0.40000 & 0.016667 \\
    1.7500 & 0.36253 & 0.36364 & 0.001110 \\
    2.0000 & 0.33333 & 0.33333 &  \\
    \hline
  \end{tabularx}
  }
  \caption{Approximation of 4a with n=3 using one Newton Iteration}
  \label{tab:4a}
\end{table}

\section{\S~11.5}
\subsection{2$^4$}
Use the Piecewise Linear Algorithm to approximate the solution to the
boundary-value problem 
$$ -\ddt{x}(xy')+4y=4x^2-8x+1,\qquad 0\leq x\leq1,\qquad
y(0)=y(1)=0 $$ using $x_0=0$, $x_1=0.4$, $x_2=0.8$, $x_3=1$. Compare
your results to the actual solution $y(x)=x^2-x$. Explicitly write the
finite-element equations. Solve and compare at nodes. Note $\int_0^1\phi_1 f
\dd{x} = -0.5813$ and $\int_0^1\phi_2 f \dd{x} = -0.7960$.
\pagebreak
\section{Minilab}
\subsection{b}
The maximum temperature occurs at approximately $(-0.12, 243.6)$.
\begin{table}[H]
  \centering
  \begin{tabularx}{.4\textwidth}{*2{>{\centering\arraybackslash}X}}
    \hline
    $x_i$ & $y_i$ \\
    \hline
    -2.00000 &   0.00000 \\  
    -1.50000 &  88.54167 \\  
    -1.00000 & 177.08333 \\  
    -0.50000 & 265.62500 \\  
    0.00000 & 291.66667 \\  
    0.50000 & 242.18750 \\  
    1.00000 & 161.45833 \\  
    1.50000 & 80.72917 \\  
    2.00000 & 0.00000 \\  
    \hline
  \end{tabularx}
  \caption{Minilab data with $\gamma=50$ and $\beta=0$}
  \label{tab:mini-b}
\end{table}

\subsection{c}
The optimal laser intensity parameter $\gamma$ I obtained was $500$,
and the optimal cooling air velocity parameter $\beta$ I obtined was
$30$. This yielded the data in
\begin{table}[H]
  \centering
  \begin{tabularx}{.4\textwidth}{*2{>{\centering\arraybackslash}X}}
    \hline
    $x_i$ & $y_i$ \\
    \hline
    -2.00000 & 0.00000 \\
    -1.50000 & 25.44353 \\
    -1.00000 & -56.42578 \\
    -0.50000 & 99.69119 \\
    0.00000 & 500.32559 \\
    0.50000 & 106.89279 \\
    1.00000 & -52.04131 \\
    1.50000 & 21.15129 \\
    2.00000 & 0.00000 \\
    \hline
  \end{tabularx}
  \caption{Minilab data with $\gamma=500$ and $\beta=30$}
  \label{tab:mini-c}
\end{table}

\pagebreak
\section{Code}
\lstinputlisting[language=Octave,label=lst:linearfe,caption=\texttt{linfindiff.m}]{linfindiff.m}
\lstinputlisting[language=Octave,label=lst:nonlinearfe,caption=\texttt{nonlinfindiff.m}]{nonlinfindiff.m}
\end{document}