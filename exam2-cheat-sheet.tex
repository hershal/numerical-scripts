\documentclass[8pt]{article}

\usepackage{mathtools}
\usepackage{amsfonts}
\usepackage{tabularx}
\usepackage{mathabx}
\usepackage{enumitem}
\usepackage{arydshln}
\usepackage{caption}
\usepackage{multicol}
\usepackage{sectsty}
\usepackage{extsizes}
\usepackage{fancyhdr}
\usepackage{enumerate}
\usepackage[margin=1.5cm]{geometry}

\pagestyle{fancyplain}

% Try very hard not to break relational and binary operators (equations)
\relpenalty=9999
\binoppenalty=9999

\begin{document}

\lhead{Hershal Bhave \quad\tiny{hb6279}}
\rhead{M368K Spring 2013 Midterm 2 Cheat Sheet (Dr. Gonzalez)}

\begin{multicols}{2}
  \begin{description}
  % \item[Definition of Eigenvalue] $p(\lambda) = \mathrm{det}(A-\lambda I)$
  \item[Ger\^{s}gorin Circle Theorem]
    \footnote{$a_{ij}$ is the $i$-th row and $j$-th column of the matrix A.}
    \footnote{The eigenvalues fall in the union of the circles. Remember $\rho(A)=\text{max}|\lambda_i|$}
    $$R^i=\left\{z\in\mathbb{C} \;\middle|\; |z-a_{jj}|\leq\sum_{j=1,j\neq i}^{n}|a_{ij}|\right\}$$
  \item[Finding Eigenvalues] Solve $\mathrm{det}(A-\lambda I)=0$
  \item[Finding Eigenvectors] Solve $(A-\lambda_iI)x=0$. The
    eigenvectors are linearly independent if $\mathrm{det}(A)\neq 0$
  \item[Orthogonal Vectors]
    $(\mathbf{v}^{(i)})^\top\mathbf{v}^{(j)}=0, \;\forall\; i\neq j$
  \item[Orthonormal Vectors]
    \footnote{Simply normalize the orthogonal set to become orthonormal}
    $(\mathbf{v}^{(i)})^\top\mathbf{v}^{(i)}=1, \;\forall\; i=1,\dots,n$ and above
  \item[Orthogonal Matrices]
    A matrix whose columns form an orthonormal set in $\mathbb{R}^n$
  \item[Positive-Definiteness] A symmetric matrix $A$ is
    positive-definite $\iff$ all its eigenvalues are positive
  \item[Invertible/Orthogonal Matrix Properties] \hfill
    \begin{enumerate}[i]
    \item Orthogonal $Q$ is invertible if $Q^{-1}=Q^\top$
    \item Invertible $Q$ is orthogonal if $Q^{-1}=Q^\top$
    \item $\forall\mathbf{x} \in \mathbb{R}^n,\|Q\mathbf{x}\|_2=\|\mathbf{x}\|_2$
    \item $Q^{-1}Q=Q^\top Q=I$
    \end{enumerate}
  \item[Similar Matrices] $A$ and $D$ similar if $\exists\,S\;|\;A=S^{-1}DS$
    \begin{enumerate}[i]
    \item $A$ and $D$ are similar with $A = S^{−1}DS$, where the
      columns of $S$ consist of the eigenvectors, and the $i$th
      diagonal element of $D$ is the eigenvalue of $A$ that
      corresponds to the $i$th column of $S$.
    \item An $n\times n$ matrix $A$ that has $n$ distinct eigenvalues is
      similar to a diagonal matrix.
    \item Diagonalizable matrices exist with $A=S^{-1}DS$ or equivalently $D=SAS^{-1}$.
    \end{enumerate}
  \item[Power Method]
    \footnote{\label{ft:powmtd_g}$\mathbf{g}^{(k)} = \mathbf{e}_{p_k}$ where $p_k$ is an array
      index: $|\mathbf{x}^{(k)}_{p_k}| = \|\mathbf{x}^{(k)}\|_\infty$}
     Rate of convergence is  $O(|\lambda_2/\lambda_1|^{m})$
    $$x^{(0)} \neq 0\text{ given, set } \mathbf{y}^{(0)}=\mathbf{x}^{(0)}$$
    \begin{equation*}
      \left\{
    \begin{aligned}
        \mathbf{x}^{(k)} &= \mathbf{y}^{(k-1)}/\|\mathbf{y}^{(k-1)}\|_\infty \\
        \mathbf{y}^{(k)} &= A\mathbf{x}^{(k)} \\
        \boldsymbol{\lambda}^{(k)} &= (\mathbf{g}^{(k)} \cdot \mathbf{y}^{(k)})/(\mathbf{g}^{(k)} \cdot \mathbf{x}^{(k)}) \\
        &=\mathbf{y}^{(k)}_{p_k}/\mathbf{x}^{(k)}_{p_k} \\
      \end{aligned}
      \right.
    \end{equation*}

  \item[Symmetric Power Method] \footnote{\label{ft:sympowmtd_g}$\mathbf{g}^{(k)} =
      \mathbf{x}^{(k)}$ where $\mathbf{g}^{(k)}\cdot\mathbf{x}^{(k)}=\|\mathbf{x}^{(k)}\|_2=1$ when $k \geq 1$} 
    Used when A is symmetric. Rate of convergence is $O(|\lambda_2/\lambda_1|^{2m})$. It is not
    known whether or not A has a single dominant eigenvalue nor how $\mathbf{x}^{(0)}$ should be chosen. 
    $$\mathbf{x}^{(0)} \neq 0 \text{ given, set } \mathbf{y}^{(0)}=\mathbf{x}^{(0)}$$
    \begin{equation*}
      \left\{
    \begin{aligned}
        \mathbf{x}^{(k)} &= \mathbf{y}^{(k-1)}/\|\mathbf{y}^{(k-1)}\|_2 \\
        \mathbf{y}^{(k)} &= A\mathbf{x}^{(k)} \\
        \boldsymbol{\lambda}^{(k)} &= (\mathbf{g}^{(k)} \cdot \mathbf{y}^{(k)})/(\mathbf{g}^{(k)} \cdot \mathbf{x}^{(k)}) \\
        &=\mathbf{y}^{(k)}_{p_k}/\mathbf{x}^{(k)}_{p_k} \\
      \end{aligned}
      \right.
    \end{equation*}
  \item[Inverse Power Method] Gives faster convergence. Used to determine the eigenvalue that is
    closest to $q$. A combined method to find all eigenvalues of a matrix is do deflate the original
    matrix, find the deflated matrix's eigenvalue through General Power Method, then use that eigenvalue
    as an initial guess to the Inverse Power Method to ensure we have found an eigenvalue of the
    original matrix, not the deflated matrix.
    % \footnote{\label{ft:invpowmtd_p}$|\mathbf{y}^{(k)}_{p_k}| = \|\mathbf{y}^{(k)}_{p_k}\|_\infty$}
    $$\mathbf{x}^{(0)} \neq 0 \text{ given, set } \mathbf{y}^{(0)}=\mathbf{x}^{(0)}$$
    \begin{equation*}
      \left\{
    \begin{aligned}
        \mathbf{x}^{(k)} &= \mathbf{y}^{(k)}/\|\mathbf{y}^{(k)}\|_\infty \\
        \mathbf{y}^{(k)} &= (A-qI)^{-1}\mathbf{x}^{(k-1)} \\
        \boldsymbol{\lambda}^{(k)} &= \frac{(\mathbf{x}^{(k)})^\top A\mathbf{x}^{(k)}}{(\mathbf{x}^{(k)})^\top\mathbf{x}^{(k)}}\\
        &=\frac{(\mathbf{x}^{(k)})^\top\mathbf{x}^{(k)}}{\|\mathbf{x}^{(k)}\|^2_2}\\
      \end{aligned}
      \right.
    \end{equation*}
  \item[Wielandt Deflation] Deleting the $i$th row and column of $B$:
    \begin{equation*}
        B=A-\lambda_1\mathbf{v}^{(1)}\mathbf{x}^\top 
        \text{ where } \mathbf{x}=\frac{1}{\lambda_1 v^{(1)}_i}(a_{i1}, a_{i2}, \ldots a_{in}) 
        \text{ }
        \footnote{$a_{i1},\ldots,a_{in}$ are entries in the $i$th row of $A$}
      \end{equation*}
    \item[Householder Transformation] Reduces a symmetric matrix to a similar tridiagonal matrix. Repeat for $k=1,\ldots,n-2$:
      \begin{equation*}
        \begin{aligned}
          q&=\sum^{n}_{j=k+1}(a_{j,k})^2 \\
          \alpha&=-\mathrm{sign}(a_{k+1,k})\sqrt{q} \\
          r&=\sqrt{\frac{1}{2}\alpha^2-\frac{1}{2}\alpha a^{(k)}_{k+1,k}} \\
          w^{(k)}_j&=\frac{a^{(k)}_{j,k}}{2r} \text{ for each } j=k+2,\ldots,n \\
          P^{(k)}&=I-2w^{(k)}*(w^{(k)})^\top \\
          A^{(k+1)}&=P^{(k)}A^{(k)}P^{(k)}
        \end{aligned}
      \end{equation*}
    \item[QR Algorithm] Finds all the eigenvalues of a symmetric, tridiagonal matrix (use Householder's).
  \end{description}
\end{multicols}
\end{document}