\documentclass[8pt]{article}

\usepackage{mathtools}
\usepackage{amsfonts}
\usepackage{tabularx}
\usepackage{mathabx}
\usepackage{enumitem}
\usepackage{arydshln}
\usepackage{caption}
\usepackage{multicol}
\usepackage{sectsty}
\usepackage{extsizes}
\usepackage{fancyhdr}
\usepackage{enumerate}
\usepackage[margin=1.5cm]{geometry}

\pagestyle{fancyplain}

% Try very hard not to break relational and binary operators (equations)
\relpenalty=9999
\binoppenalty=9999

\begin{document}

\lhead{Hershal Bhave \quad\tiny{hb6279}}
\rhead{M368K Spring 2013 Midterm 2 Cheat Sheet (Dr. Gonzalez)}

\begin{multicols}{2}
  \section{\S 9.1, \S 9.2}
  \begin{description}
  % \item[Definition of Eigenvalue] $p(\lambda) = \mathrm{det}(A-\lambda I)$
  \item[Ger\^{s}gorin Circle Theorem]
    \footnote{$a_{ij}$ is the $i$-th row and $j$-th column of the matrix A.}
    \footnote{The eigenvalues fall in the union of the circles. Remember $\rho(A)=\text{max}|\lambda_i|$}
    $$R^i=\left\{z\in\mathbb{C} \;\middle|\; |z-a_{jj}|\leq\sum_{j=1,j\neq i}^{n}|a_{ij}|\right\}$$ 
  \item[Finding Eigenvalues] Solve $\mathrm{det}(A-\lambda I)=0$
  \item[Finding Eigenvectors] Solve $(A-\lambda_iI)x=0$. The
    eigenvectors are linearly independent if $\mathrm{det}(A)\neq 0$
  \item[Orthogonal Vectors]
    $(\mathbf{v}^{(i)})^\top\mathbf{v}^{(j)}=0, \;\forall\; i\neq j$
  \item[Orthonormal Vectors]
    \footnote{Simply normalize the orthogonal set to become orthonormal}
    $(\mathbf{v}^{(i)})^\top\mathbf{v}^{(i)}=1, \;\forall\; i=1,\dots,n$ and above
  \item[Orthogonal Matrices]
    A matrix whose columns form an orthonormal set in $\mathbb{R}^n$
  \item[Invertible/Orthogonal Matrix Properties] \hfill
    \begin{enumerate}[i]
    \item Orthogonal $Q$ is invertible if $Q^{-1}=Q^\top$
    \item Invertible $Q$ is orthogonal if $Q^{-1}=Q^\top$
    \item $\forall\mathbf{x} \in \mathbb{R}^n,\|Q\mathbf{x}\|_2=\|\mathbf{x}\|_2$
    \item $Q^{-1}Q=Q^\top Q=I$
    \end{enumerate}
  \item[Similar Matrices] $A$ and $D$ similar if $\exists\,S\;|\;A=S^{-1}DS$
    \begin{enumerate}[i]
    \item $A$ and $D$ are similar with $A = S^{−1}DS$, where the
      columns of $S$ consist of the eigenvectors, and the $i$th
      diagonal element of $D$ is the eigenvalue of $A$ that
      corresponds to the $i$th column of $S$.
    \item An $n\times n$ matrix $A$ that has $n$ distinct eigenvalues is
      similar to a diagonal matrix.
    \item Diagonalizable matrices exist with $A=S^{-1}DS$ or equivalently $D=SAS^{-1}$.
    \end{enumerate}
  \item[Positive-Definiteness] A symmetric matrix $A$ is positive-definite $\iff$ all its eigenvalues are positive
  \end{description}
\end{multicols}

\end{document}