\documentclass[8pt]{article}

\usepackage{mathtools}
\usepackage{amsfonts}
\usepackage{tabularx}
\usepackage{mathabx}
\usepackage{enumitem}
\usepackage{arydshln}
\usepackage{caption}
\usepackage{multicol}
\usepackage{sectsty}
\usepackage{extsizes}
\usepackage{fancyhdr}
\usepackage{enumerate}
\usepackage[margin=1.5cm]{geometry}
\usepackage{multirow}

\pagestyle{fancyplain}

% Try very hard not to break relational and binary operators (equations)
\relpenalty=9999
\binoppenalty=9999

\newcommand{\dd}[1]{\mathrm{d}{#1}}
\newcommand{\ddt}[1]{\frac{\dd{}}{\dd{#1}}}
\newcommand{\dddt}[1]{\frac{\dd{}^2}{\dd{#1}^2}}

\begin{document}

\lhead{Hershal Bhave \quad\tiny{hb6279}}
\rhead{M368K Spring 2013 Final Exam Cheat Sheet (Dr. Gonzalez)}

\begin{multicols}{2}
  \begin{description}
  \item[Solvability Theorem for BVPs] Suppose $f$ in the BVP
    $$y''=f(x,y,y'), \text{ for } a\leq x\leq b \text{ with } y(a)=\alpha \text{ and } y(b)=\beta$$ 
    is continuous on the set 
    $$D=\{(x,y,y') \mid a\leq x\leq b, -\infty<y<\infty, -\infty<y'<\infty\}$$
    and that the partials $f_y$ and $f_{y'}$ are also continuous on $D$. If
      \begin{itemize}
      \item $f_y(x,y,y')>0, \quad\forall (x,y,y') \in D$
      \item $\exists M>0 \in \mathbb{R} \text{ s.t. } |f_{y'}(x,y,y')|\leq M \quad\forall(x,y,y')\in D$
      \end{itemize}
      Then the BVP has a unique solution.
  % \item[Solvability Theorem for IVPs] Suppose $f$ in the BVP
  %   $$y''=f(x,y,y'), \text{ for } a\leq x\leq b \text{ with } y(a)=\alpha \text{ and } y(b)=\beta$$ 
  %   is continuous on the set 
  %   $$D=\{(x,y,y') \mid a\leq x\leq b, -\infty<y<\infty, -\infty<y'<\infty\}$$
  %   and that the partials $f_y$ and $f_{y'}$ are also continuous on $D$. If
  %     \begin{enumerate}
  %     \item $f_y(x,y,y')>0, \quad\forall (x,y,y') \in D$
  %     \item $\exists M>0 \in \mathbb{R} \text{ s.t. } |f_{y'}(x,y,y')|\leq M \quad\forall(x,y,y')\in D$
  %     \end{enumerate}
  %     Then the BVP has a unique solution.
    \item[Linear Shooting Method] For general BVPs in the form
    $$f(x,y,y') = y'' = p(x)y' + q(x)y + r(x)$$ for $a\leq x \leq b,\quad
    y(a) = \alpha,\quad y(b) = \beta$
    which satisfies the simplified Solvability Theorem for BVPs:
    \begin{itemize}
    \item $p(x), q(x), r(x)$ are continuous on $[a,b]$
    \item $q(x)>0$ on $[a,b]$
    \end{itemize}
    Then the BVP has a unique solution. We can approximate this
    solution by first solving the two IVPs for $y_1$ and $y_2$ of 
    \begin{equation*} 
      \begin{aligned}
        y_1'' &= p(x)y_1' + q(x)y_1 + r(x), &\qquad y_1(a)=\alpha,&\quad y_1'(a)=0 \\
        y_2'' &= p(x)y_2' + q(x)y_2, &\qquad y_2(a)=0,&\quad y_2'(a) = 1\\
      \end{aligned}
    \end{equation*}
    using any IVP solver and then define the solution as 
    $$ y(x) = y_1(x)+\frac{\beta - y_1(b)}{y_2(b)}y_2(x) $$
    Check: Plug $y_1''$ and $y_2''$ in above and check the BVP
    conditions.
  \item[Nonlinear Shooting Method] Similar to the linear technique,
    except we approximate the solution to the boundary value problem
    using the solutions to a sequence of IVPs involving $t$ in the
    form
    $$ y'' = p(x)y' + q(x)y + r(x), \qquad y(a)=\alpha,\quad y'(a)=t_k $$
    ensuring that $\lim_{k\rightarrow\infty}y(b,t_k)=y(b)=\beta$.
    Use either Secant-Euler or Secant-RK4 to solve within each of $M$
    subintervals for the solution until $y^{(N)}-\beta < \epsilon$
  \item[Nonlinear Shooting with Secant-Euler] Given number of iterations $N$,
    number of subintervals $M$, initial $t_0, t_1$ and $h=\frac{b-a}{M}$
    \begin{equation*}
      \begin{aligned}
        F(t_k) = y^{(N)}_{t_k}(b) - \beta, &&
        M_k = \frac{F(t_{k})-F(t_{k-1})}{t_{k}-t_{k-1}}
      \end{aligned}
    \end{equation*}
    \begin{equation*}
      \begin{aligned}
         \mathbf{w}(x) = \begin{pmatrix}
          y \\
          y'' \\
        \end{pmatrix}, && V(x,y) =
        \begin{pmatrix}
          y' \\
          y'' \\
        \end{pmatrix}, && \mathbf{w}' = V(x,\mathbf{w})
      \end{aligned}
    \end{equation*}
    % \begin{equation*}
    %   \begin{aligned}
    %     %     \text{Solve } y^{(j)}_{t_0}, &\quad j=0,\ldots,M &\quad
    %     %     F(t_0) = \gamma_0 \\
    %     %     \text{Solve } y^{(j)}_{t_1}, &\quad j=0,\ldots,M &\quad
    %     %     F(t_1) = \gamma_1 \\
    %   \end{aligned}
    % \end{equation*}
    $k=1,\ldots,N$ or $F(t_k)<\epsilon$
    \begin{equation*}
      \left\{
        \begin{array}{c}
          t_{k>2}=t_{k-1}-\frac{F(t_{k-1})}{M_{k-1}} \\
          % \text{Solve } y^{(j)}_{t_1}&, \quad j=0,\ldots,M \\
          % F(t_k) &= y^{(N)}_{t_k}(b) - \beta \\
          \mathbf{w}^{(0)} =
          \left[\begin{smallmatrix}
            0 \\ t_k
          \end{smallmatrix}\right] \\
          \mathbf{w}^{(k)} = \mathbf{w}^{(k-1)} - hV(x_k,y^{(k-1)})
        \end{array}
      \right.
    \end{equation*}

  \item[Nonlinear Shooting with Newton Iteration]
    $k=1,\ldots,N$ or $F(t_k)<\epsilon$
    \begin{equation*}
      \left\{
        \begin{array}{c}
          t_{k>1}=t_{k-1}-\frac{F(t_{k-1})}{\ddt{t}y^{(k)}(b,t_{k-1})} \\
          % \text{Solve } y^{(j)}_{t_1}&, \quad j=0,\ldots,M \\
          % F(t_k) &= y^{(N)}_{t_k}(b) - \beta \\
          \mathbf{w}^{(0)} =
          \left[\begin{smallmatrix}
            0 \\ t_k
          \end{smallmatrix}\right] \\
          \mathbf{w}^{(k)} = \mathbf{w}^{(k-1)} - hV(x_k,y^{(k-1)})
        \end{array}
      \right.
    \end{equation*}

  \item[Secant/Forward Euler's Method] Is a first-order version of Runge-Kutta. For a mesh size $n$ and
    initial guess $\mathbf{x}^{(0)}$
    \begin{equation*}
      b=-\frac{1}{n}F(x)
    \end{equation*}
    \begin{equation*}
      \left\{
        \begin{aligned}
          A &= J(\mathbf{x}^{(k-1)}) \\
          \mathbf{x}^{(k)} &= \mathbf{x}^{(k-1)}+ A^{-1}b \\
        \end{aligned}
      \right.
    \end{equation*}

  \item[Shooting Method Convergence Theorem]
    Assume:
    \begin{itemize}
    \item The conditions of the BVP Solvability Theorem are met
    \item $f(t), f'(t), f''(t)$ are continuous in the neighborhood of $t$
    \item $f'(t) \neq 0$
    \end{itemize}
    Then 
    \begin{itemize}
    \item $t_k \rightarrow t_{*,N}$ as $k\rightarrow\infty$ provided
      that the initial guess was close enough ($|t_0-t_*|<r,
      |t_1-t_*|<r$) and the IVP grid is fine enough ($N\geq R$).
    \item $t_{*,N} \rightarrow t_*$ and $y_{t_{*,N}}^{(j)} \rightarrow
      y(x_{j})$ as $N\rightarrow\infty$
        \begin{gather*}
          |t_{*,N}-t_*| \leq Ch^p \\
          \stackrel{\text{max}}{_{0\leq j\leq N}}  |y_{t_{*,N}}-y(x_j)| \leq Ch^p
        \end{gather*}
      where $p=1$ for Secant-Euler or Newton-Euler and $p=4$ for
      Runge-Kutta-4.
    \end{itemize}

  \item[Runge-Kutta] For $n$ iterations and initial guess $\mathbf{x}^{(0)}$. Does not require a
    good guess of $\mathbf{x}^{(0)}$ and converges quickly, though for RK4, it requires for linear
    systems to be solved when computing the $\mathbf{k}$ values, so $n$ steps requires solving
    $4n$ linear systems. One step may be enough to get an accurate solution.

    The generalized $\mathbf{k}_i$ values, where $\alpha_i$ are weights and $h=\frac{1}{n}$
    $$ \mathbf{k}_i = -h[J(\mathbf{x}^{(k)})+\alpha_{i-1}\mathbf{k}_{i-1})]^{-1}F(\mathbf{x}^{(0)}) $$
    For the fourth-order Runge-Kutta problem ($n=4$), the $\mathbf{k}_i$ values are as below,
    where $\alpha = (0, \frac{1}{2}, \frac{1}{2}, 1)$.
    \begin{equation*}
      \begin{aligned}
        &h = 1/n & \mathbf{b} =-hF(\mathbf{x})& \\
      \end{aligned}
    \end{equation*}
    \begin{equation*}
      \left\{
        \begin{aligned}
          \mathbf{k}_1 &= (J(\mathbf{x}^{(i)}))^{-1}\mathbf{b} \\
          \mathbf{k}_2 &= (J(\mathbf{x}^{(i)}+\frac{1}{2}\mathbf{k}_1))^{-1}\mathbf{b} \\
          \mathbf{k}_3 &= (J(\mathbf{x}^{(i)}+\frac{1}{2}\mathbf{k}_2))^{-1}\mathbf{b} \\
          \mathbf{k}_4 &= (J(\mathbf{x}^{(i)}+\mathbf{k}_3))^{-1}\mathbf{b} \\
          \mathbf{x}^{(i+1)} &= \mathbf{x}^{(i)} + \frac{1}{6}(\mathbf{k}_1+2\mathbf{k}_2+2\mathbf{k}_3+\mathbf{k}_4) \\
        \end{aligned}
      \right.
    \end{equation*}
  \item[Centered-Difference Equations for Non/Linear Problems] 
    \begin{gather*}
      y'(x_i)=\frac{1}{2h}[y(x_{i+1})-y(x_{i-1})]-\frac{h^2}{6}y'''(\eta_i) \\
      y''(x_i)=\frac{1}{h^2}[y(x_{i+1})-2y(x_i)+y(x_{i-1})]-\frac{h^2}{12}y^{(4)}(\xi_i) \\
      \frac{y(x_{i+1})-2y(x_i)+y(x_{i-1})}{h^2} = p\frac{y(x_{i+1}) - y(x_{i-1})}{2h}+qy(x_i) + r \\
    \end{gather*}
    % \pagebreak[2]
  \item[Linear Finite Difference Method] 
    For general BVPs, where \linebreak[4]
    $a\leq x \leq b,\quad y(a) = \alpha,\quad y(b) = \beta$
    $$f(x,y,y') = y'' = p(x)y' + q(x)y + r(x)$$
    The Centered-Difference Equations are rewritten in the form
    \begin{equation*}
      \left(\frac{-w_{i+1}+2w_i-w_{i-1}}{h^2}\right) +
      p(x_i)\left(\frac{w_{i+1}-w_{i-1}}{2h}\right) + q(x_i)w_i = -r(x_i)
    \end{equation*}
    Define $w_0 = \alpha$, $w_{N+1}=\beta$, solve the tridiagonal
    $N\times N$ matrix $A\mathbf{w}=\mathbf{b}$ for the $N$ interior
    nodes (subintervals).
    \begin{equation*}
      \left\{
        \begin{aligned}
          &A =
          % \begin{pmatrix}
          %   %   1-2\lambda & \lambda & 0 & \multicolumn{3}{c}{\cdots} & 0 \\
          %   %   \lambda & 1-2\lambda & \lambda & \multicolumn{2}{c}{\ddots} &
          %   %   & \multirow{2}{*}{\vdots} \\
          %   %   0 & & \multirow{2}{*}{\ddots} \\
          %   %   \vdots & & & & & & 0 \\
          %   %   0 & \multicolumn{2}{c}{\cdots} & 0 & \lambda & 1-2\lambda \\
          %   2+h^2q(x_1) & -1+\frac{h}{2}p(x_1)    & 0       & \multicolumn{2}{c}{\cdots} & 0 \\
          %   -1+\frac{h}{2}p(x_1)    & 2+h^2q(x_1) & -1+\frac{h}{2}p(x_1) & \ddots &  & \multirow{2}{*}{\vdots} \\
          %   0 & \ddots & \ddots & \ddots & \ddots & & \\
          %   \multirow{2}{*}{\vdots} & \ddots & \ddots & \ddots & \ddots & 0 \\
          %   & & \ddots & -1+\frac{h}{2}p(x_1) & 2+h^2q(x_1) & -1+\frac{h}{2}p(x_1) \\
          %   0 & \multicolumn{2}{c}{\cdots} & 0 & -1+\frac{h}{2}p(x_1) & 2+h^2q(x_1) \\
          % \end{pmatrix} \\
          \footnotesize{\begin{pmatrix}
            2+h^2q(x_1) & -1+\frac{h}{2}p(x_1)& & 0\\
            -1-\frac{h}{2}p(x_2)& \ddots & \ddots & \\
            & \ddots & \ddots & -1+\frac{h}{2}p(x_{N-1})\\
            0 & & -1-\frac{h}{2}p(x_N)& 2+h^2q(x_N)
          \end{pmatrix}},& \\
          % \mathbf{w}^{(0)} &= \sin\frac{\pi}{4}x \cdot
          % \left(1+2\cos\frac{\pi}{4}x\right)\\
          &\mathbf{w}=\begin{pmatrix} w_1 \\ w_2 \\ \vdots \\ w_{N-1} \\ w_N \end{pmatrix},
          \qquad \mathbf{b} =
          \begin{pmatrix}
            -h^2r(x_1)+\left(1+\frac{h}{2}p(x_1)\right)w_0 \\
            -h^2r(x_2) \\
            \vdots \\
            -h^2r(x_{N-1}) \\
            -h^2r(x_N)+\left(1+\frac{h}{2}p(x_N)\right)w_{N+1} \\
          \end{pmatrix}& \\
        \end{aligned}
      \right.
    \end{equation*}
  \item[Linear Finite-Difference Convergence Theorem]
    The tridiagonal linear system has a unique solution provided that
    $h<2/L$ where $L=\stackrel{\text{max}}{_{a\leq
        x\leq b}}|p(x)|$. Establish $y^{(4)}$ is continuous for
    truncation error $O(h^2)$.
  \item[Nonlinear Finite Difference Method]     
    For general BVPs, where \linebreak[4]
    $a\leq x \leq b,\quad y(a) = \alpha,\quad y(b) = \beta$
    $$f(x,y,y') = y'' = p(x)y' + q(x)y + r(x)$$
    To guarantee a unique solution, assume that $f$ satisfies:
    \begin{itemize}
    \item $f$ and its partials $f_y$ and $f_{y'}$ are continuous on
      $$D=\{(x,y,y') \mid a\leq x\leq b, -\infty<y<\infty, -\infty<y'<\infty\}$$
    \item $f_y(x,y,y')\geq\delta$ on $D$, for some $\delta>0$
    \item Constants $k$ and $L$ exist such that
      \begin{equation*}
        \begin{array}{cc}
          k=\stackrel{\text{max}}{_{(x,y,y')\in D}}|f_y(x,y,y')|, &
          L=\stackrel{\text{max}}{_{(x,y,y')\in D}}|f_{y'}(x,y,y')| 
        \end{array}
      \end{equation*}
    \end{itemize}
    The Centered Difference Method uses $w_0 = \alpha$, $w_{N+1} =
    \beta$:
    $$-\frac{w_{i+1}-2w_i+w_{i-1}}{h^2} +
    f\left(x_i,w_i,\frac{w_{i+1}-w_{i-1}}{2h}\right) = 0$$ 
    for $i=1,\ldots,N$.  Multiply both sides by $h^2$ and we can use
    Newton's Method to approximate the solution, so the next
    iterations of the method will converge to the solution provided
    that the initial approximation of $w_i^{(0)}$ is close to the
    solution and that the Jacobian matrix is nonsingular. Solve for
    $\mathbf{v}$ and accumulate with $\mathbf{w}$ to obtain the
    solution. The Jacobian matrix is tridiagonal, and the method
    converges of order $O(h^2)$.
    \begin{equation*}
      \left\{
        \begin{aligned}
          &J(w_1,\ldots,w_N)_{i,j}^{(k)}= \\
          & \left\{
            \begin{array}{ccc}
              -1+\frac{h}{2}f_{y'}\left(x_i,w_i,\frac{w_{i+1}-w{i-1}}{2h}\right), & i=j-1, & j=2,\ldots,N \\
              2+h^2f_y\left(x_i,w_i,\frac{w_{i+1}-w_{i-1}}{2h}\right), & i=j & j=1,\ldots,N \\
              -1-\frac{h}{2}f_{y'}\left(x_i,w_i,\frac{w_{i+1}-w{i-1}}{2h}\right), & i=j-1, & j=2,\ldots,N \\
            \end{array}
          \right.\\
          &J^{(k)}\mathbf{v}^{(k)} = \begin{pmatrix} (-\frac{w_{i+1}-2w_i+w_{i-1}}{h^2} +
            f\left(x_i,w_i,\frac{w_{i+1}-w_{i-1}}{2h}\right) \end{pmatrix}\\
          &\mathbf{w}^{(k)} = \mathbf{w}^{(k-1)} + \mathbf{v}^{(k)}
        \end{aligned}
      \right.
    \end{equation*}
  \item[Newton's Method] Exhibits quadratic convergence; requires
    continuous $g$ derivatives and that $A(\mathbf{x}) =
    J(\mathbf{x})$ is nonsingular around the radius of the
    solution.  Given an initial guess $\mathbf{x}^{(0)}$:
    \begin{align*}
      \mathbf{x}^{(k)}=G(\mathbf{x}^{(k-1)})&=\mathbf{x}^{(k-1)}-A(\mathbf{x}^{(k-1)})^{-1}F(\mathbf{x}^{(k-1)}) \\
      &=\mathbf{x}^{(k-1)}-J(\mathbf{x}^{(k-1)})^{-1}F(\mathbf{x}^{(k-1)}) \\
    \end{align*}
    A weakness in Newton's is the requirement to compute and invert $J(\mathbf{x})$,
    avoided by finding a $\mathbf{y}$ such that
    $$J(\mathbf{x}^{(k-1)})\mathbf{y}=-F(\mathbf{x}^{(k-1)})$$
    Then $\mathbf{x}^{(k)}=\mathbf{x}^{(k-1)}+\mathbf{y}$. Newton's
    Method requires $n^2+n$ functional evaluations, $n^2$ for the
    Jacobian matrix, $n$ for the evaluation of $F$, and
    $O(n^3)$ arithmetic operations to solve the linear system.
  \item[Piecewise Basis Functions]
    We must first partition the domain such that
    $a=x_0<x_1<\cdots<x_n<x_{n+1}=b$ for $n$ subintervals. Pay
    attention to the value of $x$; remember these equations are
    piecewise. Remember this especially when integrating.
    \begin{equation*}
      \begin{aligned}
        \phi_i(x) &= \left\{
          \begin{array}{cc}
            0, & a\leq x\leq x_{i-1} \\
            \frac{1}{h_{i-1}}(x-x_{i-1}) & x_{i-1}< x\leq x_i \\
            \frac{1}{h_{i}}(x-x_{i+1}) & x_{i}< x\leq x_{i+1} \\
            0, & x_{i+1}< x\leq b
          \end{array}
        \right. \\
        \phi_i'(x) &= \left\{
          \begin{array}{cc}
            0, & a\leq x\leq x_{i-1} \\
            \frac{1}{h_{i-1}} & x_{i-1}< x\leq x_i \\
            -\frac{1}{h_{i}} & x_{i}< x\leq x_{i+1} \\
            0, & x_{i+1}< x\leq b
          \end{array}
        \right.
      \end{aligned}
    \end{equation*}
  \item[Piecewise Linear Finite Element Method] The matrix $A$ is
    symmetric, tridiagonal, and positive-definite (given $p(x)\geq
    p_{min}>0$, so the linear system is stable with respect to
    roundoff error. We have $|\phi(x)-y(x)|=O(h^2)$ convergence for
    each $x$ in $[a,b]$ due to the first-degree interpolating
    polynomial $y^*(x)$.

    \begin{equation*}
      \left\{
        \begin{aligned}
          a_{ij} &=
          \int_a^b[p(x)\phi'_i(x)\phi'_j(x)+q(x)\phi'_i(x)\phi'_j(x)]\dd{x} \\
          b_i &= \int_a^bf(x)\phi_i(x)\dd{x} \\
          A\mathbf{c} &= \mathbf{b} \\
          y^*(x) &= \sum_{i=1}^{n}c_i\phi_i(x) \\
        \end{aligned}
      \right.
    \end{equation*}
    or
    \begin{equation*}
      \left\{
        \begin{aligned}
          Q_{1,i}&=\left(\frac{1}{h_{i}}\right)^2\int_{x_i}^{x_{i+1}}(x_{i+1}-x)(x-x_{i})q(x)\dd{x},& i=1,\ldots,n-1 \\
          Q_{2,i}&=\left(\frac{1}{h_{i-1}}\right)^2\int_{x_{i-1}}^{x_{i}}(x-x_{i-1})^2q(x)\dd{x},& i=1,\ldots,n \\
          Q_{3,i}&=\left(\frac{1}{h_{i}}\right)^2\int_{x_i}^{x_{i+1}}(x_{i+1}-x)^2q(x)\dd{x},& i=1,\ldots,n \\
          Q_{4,i}&=\left(\frac{1}{h_{i-1}}\right)^2\int_{x_{i-1}}^{x_{i}}p(x)\dd{x},& i=1,\ldots,n+1 \\
          Q_{5,i}&=\frac{1}{h_{i-1}}\int_{x_{i-1}}^{x_{i}}(x-x_{i-1})f(x)\dd{x},& i=1,\ldots,n \\
          Q_{6,i}&=\frac{1}{h_{i}}\int_{x_i}^{x_{i+1}}(x_{i+1}-x)f(x)\dd{x},& i=1,\ldots,n \\
          a_{i,i} &= Q_{4,i}+Q_{4,i+1}+Q_{2,i}+Q_{3,i},& i=1,\ldots,n \\
          a_{i,i+1} &= -Q_{4,i+1}+Q_{1,i},& i=1,\ldots,n-1 \\
          a_{i,i-1} &= -Q_{4,i}+Q_{1,i-1},& i=2,\ldots,n \\
          b_{i} &= Q_{5,i}+Q_{6,i},& i=1,\ldots,n \\
          A\mathbf{c} &= \mathbf{b} \\
          y^*(x) &= \sum_{i=1}^{n}c_i\phi_i(x) \\
        \end{aligned}
      \right.
    \end{equation*}
  \item[Poisson Equation Finite-Difference Method]
    In general form:
    \begin{equation*}
      \begin{aligned}
        \nabla^2u(x,y)&\equiv\frac{\partial^2u}{\partial
          x^2}(x,y)+\frac{\partial^2u}{\partial y^2}=f(x,y) \\
        R &= \{(x,y) \mid a<x<b,\;c<y<d\} \\
        u(x,y) &= g(x,y),\quad (x,y) \in S
      \end{aligned}
    \end{equation*}
    $S$ is a boundary of $R$. If $f$ and $g$ are continuous then there
    is a unique solution.  We first choose a grid of step sizes
    $h=(b-a)/n$ and $k=(d-c)/m$; $n$ and $m$ are the number of steps
    for the $x$ and $y$ axis. $x_i=a+ih$ and $y_i=c+ik$ are gridlines
    whose intersections form mesh points. The Centered-Difference
    Formulas:
    \begin{equation*}
      \begin{aligned}
        \frac{\partial^2u}{\partial x^2}(x_i,y_j) &=
        \frac{u(x_{i+1},y_{j})-2u(x_{i},y_{j}) + u(x_{i-1},y_{j})}{h^2} -
        \frac{h^2}{12}\frac{\partial^4u}{\partial x^4}(\xi_i,y_j) \\
        \frac{\partial^2u}{\partial y^2}(x_i,y_j) &=
        \frac{u(x_{i},y_{j+1})-2u(x_{i},y_{j}) + u(x_{i},y_{j-1})}{k^2} -
        \frac{k^2}{12}\frac{\partial^4u}{\partial y^4}(x_i,\eta_j) \\
      \end{aligned}
    \end{equation*}
    $\xi_i \in (x_{i-1},x_{i+1})$; $\eta_j \in (y_{j-1},y_{j+1})$.
    Difference-Equation form:
    \begin{equation*}
      \scriptsize{
        \begin{array}{c}
          2\left[\left(\frac{h}{k}\right)^2+1\right]w_{i,j}-(w_{i+1,j}+w_{i-1,j})
          - \left(\frac{h}{k}\right)^2(w_{i,j+1}+w_{i,j-1})=-h^2f(x_i,y_j)
        \end{array}}
    \end{equation*}
    $\lambda = \left(\frac{h}{k}\right)^2$. Label the grid points
    $P_l=(x_i,y_j)$ and $w_l=w_{i,j}$, where $l=i+(m-1-j)(n-1)$ for
    $i=1,\ldots,n-1$ and $j=1,\ldots,m-1$ to obtain equations at each
    point $P_i$. Set the RHS of those equations to the adjacent
    boundary conditions. Put the $P_i$ equations in a matrix and solve
    for $w_i$. Gauss-Seidel is used to solve the matrix: it is
    Symmetric Positive-Definite and diagonally dominant (Symmetric
    Block Tridiagonal).
  \item[Poisson Equation Solving Methods] Iterative methods should be
    used for large systems, specifically SOR.
    % , whose optimal $w$ is 
    % $$w=\frac{2}{1+\sqrt{1-[\rho(B)]^2}}=\frac{4}{2+\sqrt{4-[\cos(\pi/m)+\cos(\pi/n)]^2}}$$
    Otherwise direct techniques usually work better (less roundoff).
    \pagebreak[4]
  \item[Forward Difference Method] For Parabolic (heat/diffusion) PDEs:
    \begin{equation*}
      \begin{aligned}
        &\frac{\partial u}{\partial t}(x,t) =
        \alpha^2\frac{\partial^2u}{\partial x^2}(x,t), \qquad
        0<x<l,\quad t>0& \\
        &u(0,t)=u(l,t)=0,\quad t>0; \quad u(x,0)=f(x),\quad 0\leq x\leq
        l& \\
      \end{aligned}
    \end{equation*}
    The discrete equation is
    $$w_{i,j+1}=(1-2\lambda)w_{i,j}+\lambda(w_{i+1,j}+w_{i-1,j})$$
    Where $w_{i,0}=f(x_i)$. The system can be written in
    matrix form:
    \begin{equation*}
      \begin{aligned}
        A&=
        \begin{pmatrix}
            1-2\lambda & \lambda & & 0\\
            \lambda& \ddots & \ddots & \\
            & \ddots & \ddots & \lambda\\
            0 & & \lambda & 1-2\lambda
          \end{pmatrix} \\
      \end{aligned}
    \end{equation*}
    where $\lambda = k\left(\frac{\alpha}{h}\right)^2,\;h=\Delta
    x,\;k=\Delta t$. Using the matrix form, further approximations of
    $\mathbf{w}$ can be obtained by multiplying:
    $$ \mathbf{w}^{(j)} = A\mathbf{w}^{(j-1)} $$
    The Forward Difference method is conditionally stable: for
    staibilty, $\rho(A)\leq1$ because of initial error propogation in
    the explicit nature of the difference method:
    $$\mathbf{w}^{(1)}=A(\mathbf{w}^{(0)}+\mathbf{e}^{(0)}) =
    A\mathbf{w}^{(0)} + A\mathbf{e}^{(0)}$$ 
    Choose $h$, $k$ such that $\lambda\leq\frac{1}{2}$ to ensure
    $O(k+h^2)$ convergence. 
  \item[Backward Difference Method] Same Parabolic (heat/diffusion)
    PDE. The discrete equation is
    $$w_{i,j-1}=(1+2\lambda)w_{i,j}-\lambda(w_{i+1,j}+w_{i-1,j})$$
    Where $w_{i,0}=f(x_i)$. The system can be written in matrix form:
    \begin{equation*}
      \begin{aligned}
        A&=
        \begin{pmatrix}
            1+2\lambda & -\lambda & & 0\\
            -\lambda& \ddots & \ddots & \\
            & \ddots & \ddots & -\lambda\\
            0 & & -\lambda & 1+2\lambda
          \end{pmatrix} \\
      \end{aligned}
    \end{equation*}
    where $\lambda = k\left(\frac{\alpha}{h}\right)^2\;h=\Delta
    x,\;k=\Delta t$. Using the matrix form, further approximations of
    $\mathbf{w}$ can be obtained by solving:
    $$ \mathbf{w}^{(j-1)} = A\mathbf{w}^{(j)} $$
    The Backward Difference method is unconditionally stable: The
    implicit-difference nature of the Backward Difference
    method. Since $\lambda>0$, $A$ is positive-definite, strictly
    diagonally dominant, and tridiagonal. Since $\rho(A^{-1})<1$ (since
    the eigenvalues of $A^{-1}$ are reciprocals of those from $A$),
    $\lim_{n\rightarrow\infty}(A^{-1})^n\mathbf{e}^{0}=0$. The rate of
    convergence and local truncation error is $O(k+h^2)$.
  \item[Crank-Nicolson Method] The discrete equation, using an
    averaged-difference method:
    \begin{equation*}
      \begin{array}{c}
        \frac{w_{i,j+1}-w_{i,j}}{k}=\frac{\alpha^2}{2} \left[\frac{w_{i+1,j}-2w_{i,j}+w_{i-1,j}}{h^2} + \frac{w_{i+1,j}-2w_{i,j}+w_{i-1,j}}{h^2}\right]
      \end{array}
    \end{equation*}
    In matrix form this is written as
    \begin{equation*}
      \begin{aligned}
        A&=
        \begin{pmatrix}
          (1+\lambda) & -\frac{\lambda}{2} & & 0\\
          -\frac{\lambda}{2} & \ddots & \ddots & \\
          & \ddots & \ddots & -\frac{\lambda}{2}\\
          0 & & -\frac{\lambda}{2} & (1+\lambda)
        \end{pmatrix} \\
        B&=
        \begin{pmatrix}
          (1-\lambda) & \frac{\lambda}{2} & & 0\\
          \frac{\lambda}{2} & \ddots & \ddots & \\
          & \ddots & \ddots & \frac{\lambda}{2}\\
          0 & & \frac{\lambda}{2} & (1-\lambda)
        \end{pmatrix} \\
      \end{aligned}
    \end{equation*}
    where $\lambda = k\left(\frac{\alpha}{h}\right)^2,\;h=\Delta
    x,\;k=\Delta t$. Using the matrix form, further approximations of
    $\mathbf{w}$ can be obtained by solving:
    $$ A\mathbf{w}^{(j+1)} = B\mathbf{w}^{(j)} $$
  \item[Central-Difference Method] For Hyperbolic (wave) PDEs:
    \begin{equation*}
      \begin{aligned}
        &\frac{\partial^2u}{\partial t^2}(x,t) =
        \alpha^2\frac{\partial^2u}{\partial x^2}(x,t),
        &0<x<l, \qquad t>0; \\
        &u(0,t)=u(l,t)=0,\quad &t>0; \\
        &u(x,0)=f(x),\qquad \frac{\partial u}{\partial t}(x,0)=g(x), \quad &0\leq x\leq
        l& \\ 
      \end{aligned}
    \end{equation*}
    The discrete equation is 
    $$ w_{i,j+1} = 2(1-\lambda^2)w{i,j} + \lambda^2(w_{i+1,j}+w_{i-1,j})-w_{i,j}-1 $$
    where $w_{0,j}=w_{m,j}=0$, $w_{i,0}=f(x_i)$, and 
    \begin{equation*}
      \begin{aligned}
        \mathbf{w}_{i,0} &= f(x_i) \\
        \mathbf{w}_{i,1} &=
        (1-\lambda^2)f(x_i)+\frac{\lambda^2}{2}f(x_{i+1}) +
        \frac{\lambda^2}{2}f(x_{i-1}) + kg(x_i)
      \end{aligned}
    \end{equation*}
    In matrix form this is written as
    \begin{equation*}
      \begin{aligned}
        A&=
        \begin{pmatrix}
            2(1-\lambda^2) & \lambda^2 & & 0\\
            -\lambda& \ddots & \ddots & \\
            & \ddots & \ddots & -\lambda\\
            0 & & \lambda^2 & 2(1-\lambda^2)
          \end{pmatrix} \\
      \end{aligned}
    \end{equation*}
    where $\lambda=\alpha\frac{k}{h},\;h=\Delta x,\;k=\Delta
    t$. Further approximations of $\mathbf{w}$ can be obtained by
    computing:
    $$ \mathbf{w}^{(j+1)} = A \mathbf{w}^{(j)} - \mathbf{w}^{(j-1)}$$
    Since this method is explicit, it is conditionally stable: we must
    choose $h$, $k$ such that $\lambda\leq1$ to ensure $O(h^2+k^2)$
    convergence.
  \end{description}
\end{multicols}
\end{document} 